%!TEX program = xelatex
\documentclass[aspectratio=169]{beamer}
\usepackage[UTF8]{ctex} % Use default font set
\usepackage{hyperref}

% other packages
\usepackage{latexsym,amsmath,xcolor,multicol,booktabs,calligra}
\usepackage{siunitx} % For \SI command
\usepackage{graphicx,pstricks,listings,stackengine}

\author{Yu Shu \& Chihao Shi}
\title{力学A(PHYS1001A.04):第三次习题课}
\subtitle{Course NOT easy: The Survival Guide 1+2}
\institute{School of Physics, USTC}
\date{Nov.3, 2025}
\usepackage{USTC} 

% defs
\def\cmd#1{\texttt{\color{red}\footnotesize $\backslash$#1}}
\def\env#1{\texttt{\color{blue}\footnotesize #1}}
\definecolor{deepblue}{rgb}{0,0,0.5}
\definecolor{deepred}{rgb}{0.6,0,0}
\definecolor{deepgreen}{rgb}{0,0.5,0}
\definecolor{halfgray}{gray}{0.55}

\lstset{
    basicstyle=\ttfamily\small,
    keywordstyle=\bfseries\color{deepblue},
    emphstyle=\ttfamily\color{deepred},    % Custom highlighting style
    stringstyle=\color{deepgreen},
    numbers=left,
    numberstyle=\small\color{halfgray},
    rulesepcolor=\color{red!20!green!20!blue!20},
    frame=shadowbox,
}

\begin{document}

\begin{frame}
    \titlepage
    \begin{figure}[htpb]
        \begin{center}
            \includegraphics[width=0.15\linewidth]{pic/ustc_logo_fig-eps-converted-to.pdf}
        \end{center}
    \end{figure}
\end{frame}

\begin{frame}
    \tableofcontents[sectionstyle=show,subsectionstyle=show/shaded/hide,subsubsectionstyle=show/shaded/hide]
\end{frame}

\section{内容回顾与补充拓展}

\subsection{非惯性系}

\begin{frame}
    惯性系:物体满足牛顿第一定律,即在不受力的情况下,速度的大小与方向不变. 同时,在此参考系下,牛顿第二定律有形式$\vec{F}=m\vec{a}$.

    非惯性系:相对于惯性系作变速运动的参考系
\end{frame}

\begin{frame}{平动加速参考系}
    平移惯性力:$\vec{f}_i=-m\vec{a}_0$

    表现力:$\vec{F}_{eff}=\vec{F}+\vec{f}_i$

    平动加速参考系下的牛顿第二定律:$\vec{F}_{eff}=m\vec{a}’$
\end{frame}

\begin{frame}
    “虚拟力”\&“真实力”
    \begin{enumerate}
        \item 不能指出是哪个物体作用;
        \item 没有反作用力;
        \item 所有质点都受力,而且惯性力与质点的位置无关,各处均匀。 其指向一律与“牵连”加速度(坐标系$K'$的加速度)相反,且正比于质量(和重力类似);
        \item 原则上讲,只要选择惯性系,就可以消除惯性力,而真实力一般不能这样来消除。
    \end{enumerate}
\end{frame}

\begin{frame}{补充拓展:等效原理}
    弱等效原理(Weak equivalence principle):$m_i=m_g$,$m_g$为引力质量,$m_i$为惯性质量

    爱因斯坦表述(Einstein equivalence principle):引力场与惯性力场等效

    强等效原理(Strong equivalence principle):在时空区域的一点内的引力场可用相应的局域惯性参考系去描述,而狭义相对论在其局域惯性参考系中完全成立

    例:狭义相对论下$T^{\mu\nu}_{,\nu}=0$,引入引力场后,广义相对论下$T^{\mu\nu}_{;\nu}=T^{\mu\nu}_{,\nu}+\Gamma^{\mu}_{\rho\nu}T^{\rho\nu}+\Gamma^{\nu}_{\rho\nu}T^{\rho\mu}=0$
\end{frame}

\begin{frame}{补充拓展:厄缶实验(Eötvös experiment)}
    \begin{figure}[htpb]
        \begin{center}
            \includegraphics[width=0.95\linewidth]{pic/1.png}
        \end{center}
    \end{figure}
\end{frame}

\begin{frame}{补充拓展:潮汐}
    潮汐现象主要来自于引力的空间不均匀性
    \begin{figure}[htpb]
        \begin{center}
            \includegraphics[width=0.95\linewidth]{pic/2.png}
        \end{center}
    \end{figure}

    \url{https://science.nasa.gov/moon/tides/}
\end{frame}

\begin{frame}{转动惯性系}
    $K'$系相对于$K$系有转动,则有:
    \begin{enumerate}
        \item $K'$中的位矢:$\vec{r}' = \vec{r} - \vec{r}_0$
        \item 单位矢量关系:$\dot{\vec{\imath}'}=\vec{\omega}\times\vec{\imath}'$,$\dot{\vec{\jmath}'}=\vec{\omega}\times\vec{\jmath}'$,$\dot{\vec{k}'}=\vec{\omega}\times\vec{k}'$
        \item 速度变换:$\vec{v}=\vec{v}_0+\vec{\omega}\times\vec{r}'+\vec{v}'$,其中,$\vec{v}'$为$K'$系中的速度向量,有$\vec{v}'=\dot{r}_x\vec{\imath}'+\dot{r}_y\vec{\jmath}'+\dot{r}_z\vec{k}'\neq\frac{d\vec{r}'}{dt}$
        \item 加速度变换:$\vec{a}=\vec{a}_0+\vec{a}'+2\vec{\omega}\times\vec{v}'+\dot{\vec{\omega}}\times\vec{r}'+\vec{\omega}\times(\vec{\omega}\times\vec{r}')$
    \end{enumerate}

    加速度变换包括以下变换项:
    \begin{enumerate}
        \item 平动惯性力:$-m\vec{a}_0$
        \item 惯性离心力:$-m\vec{\omega}\times(\vec{\omega}\times\vec{v}')$
        \item 科里奥利力:$-2m\vec{\omega}\times\vec{v}'$
        \item 切向惯性力:$-m\dot{\vec{\omega}}\times\vec{r}'$
    \end{enumerate}
\end{frame}

\begin{frame}{惯性离心力}
    一个相对于转动系$K'$静止的物体,在惯性系$K$看来,它必定绕转轴作圆周运动,其上受向心力$\vec{F}=m\vec{\omega}\times(\vec{\omega}\times\vec{v}')$。
    在转动系中此物静止不动,必须认为物体不仅受真实力$\vec{F}$作用,而且还受虚拟力$\vec{F}_c$作用, $\vec{F}_c$正好与$\vec{F}$相抵消。

    惯性离心力垂直于转轴,并指向离开转轴的方向。

    惯性离心力与物体质量成正比。离心力与物体所在位置有关,与物体在转动系中运动与否无关。

    视重:地球是一个转动参考系,表观重力为惯性离心力与引力的合力,且与纬度有关。
\end{frame}

\begin{frame}{科里奥利力}
    如果物体在转动系𝐾′中运动,则产生新的惯性力,即科里奥利力$\vec{F}_{cor}=-2m\vec{\omega}\times\vec{v}'$。

    用右手螺旋法则,将右手伸出,用四个指头指向质点相对$\vec{K}'$系的速度方向,再将四指绕向角速度方向,拇指所指方向为科里奥利力方向。

    与相对速度成正比,故只有当物体相对转动参考系运动时才能出现。

    与转动角速度的一次方成正比,而离心力与角速度的二次方成正比,故当参考系的转动角速度较小时,科里奥利力比离心力更重要。

    力的方向总是与相对速度垂直,不会改变相对速度的大小。
\end{frame}

\begin{frame}{科里奥利力}
    地球是一个转动参考系,科里奥利力在地球上的表现:
    \begin{enumerate}
        \item 地面上北半球河流冲刷右岸,火车对右轨的偏压较大。在南半球则对左岸和左轨作用大。
        \item 地球上自由落体偏东。
    \end{enumerate}

    傅科摆(Foucault pendulum)摆平面转动角速度:$\Omega=-\omega\sin\lambda$。
\end{frame}

\begin{frame}
    \begin{figure}[htpb]
        \begin{center}
            \includegraphics[width=0.95\linewidth]{pic/3.png}
        \end{center}
    \end{figure}
\end{frame}

\subsection{相对性原理和绝对时空观}

\begin{frame}
    本部分可见第一次习题课讲义P40-P47
\end{frame}

\subsection{守恒律}

\begin{frame}
    多体问题难以求解$m\ddot{\vec{r}}=\vec{F}(\vec{r},\dot{\vec{r}},t)$。

    力是必须的吗?

    守恒律是牛顿三定律的推论,是求解系统运动特征的有力工具。

    守恒律的应用范围比牛顿三定律更广泛。

    诺特定理(Noether's theorem): 每个连续对称性都有着相应的守恒定律。
\end{frame}

\subsection{动量定理}

\begin{frame}{质点动量定理}
    动量: $p = mv$

    质点运动方程: $m\frac{d^2x}{dt^2} = F$

    质点动量定理微分形式:$\vec{F}dt=d\vec{p}$

    质点动量定理积分形式:$\int_{t_1}^{t_2}\vec{F}dt=\int_{\vec{p}_1}^{\vec{p}_2}d\vec{p}=\vec{p}_2-\vec{p}_1$

    冲量: $\vec{I}=\int_{t_1}^{t_2}\vec{F}dt$

    冲量定理: $\vec{I}=\vec{p}_2-\vec{p}_1$

    力对质点的冲量等于质点动量的增加
\end{frame}

\begin{frame}{质点系动量定理}
    质点系(质点组):由相互作用的若干个质点组成的系

    内力:系统内各质点间的相互作用力

    外力:系统以外的其它物体对系统内任意一质点的作用力

    质点系动量定理: $\vec{F}_{ex}=\frac{d\vec{p}}{dt}$
\end{frame}

\begin{frame}
    原惯性系有$\vec{F}$的作用

    对于惯性系,$\vec{F}dt=\vec{F}'dt'$,则$\vec{F}'dt'=d\vec{p}'$

    对于非惯性系,则要考虑惯性力的冲量:$\vec{I}_{ext}+\vec{I}_{inert}=\vec{p}_2-\vec{p}_1$

    加速平动参考系:$d\vec{p}'=(\vec{F}-m\vec{a}_0)dt$

    加速转动参考系:$d\vec{p}'=(\vec{F}-m\vec{a}_0-2m\vec{\omega}\times\vec{r}'-m\vec{\omega}\times(\vec{\omega}\times\vec{r}')-m\dot{\vec{\omega}}\times\vec{r}')dt$
\end{frame}

\begin{frame}
    只有外力对体系的总动量变化有贡献。内力对体系的总动量变化没有贡献,但内力对动量在体系内部的分配是有作用的

    动量定理是矢量式,应用时可用沿坐标轴的分量式求解

    动量定理与牛顿定律的关系:
    \begin{enumerate}
        \item 对一个质点,牛顿定律表示的是力的瞬时效应,而动量定理表示的是力对时间的积累效果;
        \item 牛顿第二定律只适于质点,不能直接用于质点系;动量定理既适于质点又适于质点系;
        \item 牛顿第三定律不适用的地方,动量定理也不适用;
        \item 牛顿定律和动量定理都只适用于惯性系,要在非惯性系中应用动量定理,必须考虑惯性力的冲量。
    \end{enumerate}
\end{frame}

\begin{frame}
    \begin{equation}
        \vec{F}_{ex}=\frac{d\vec{p}}{dt}
    \end{equation}

    注意:
    \begin{enumerate}
        \item 只适用于惯性系;
        \item 动量守恒是矢量式,它有三个分量,各分量可以分别守恒;
        \item 在某些过程(如爆炸、碰撞)中, 体系虽受外力,但外力有限(外力≪内力),过程时间很短,外力冲量很小;而其间内力很大, 体系内每一部分的动量变化主要来自内力的冲量,外力的冲量可忽略不计,体系动量近似守恒,故可以利用动量守恒定律研究体系内部各部分间的动量再分配问题。
    \end{enumerate}
\end{frame}

\begin{frame}{质心运动定理}
    \begin{equation}
        \vec{F}_{ex}=\frac{d\vec{p}}{dt}=M_c\ddot{\vec{r}}_C=M_C\vec{a}_C
    \end{equation}

    对于单个质点,动量定理与牛二是等效的,但不适用于质点系(因为每个质点的加速度不同)

    但对质点系而言,确实存在一个特殊点(质心),可以应用牛顿第二定律
\end{frame}

\begin{frame}{质心}
    分立质点系:$\vec{r}_C=\frac{\sum_{i=1}^{N}m_i\vec{r}_i}{\sum_{i=1}^{N}m_i}$

    连续体:$\vec{r}_C=\frac{\int\vec{r}dm}{\int dm}$

    巴普斯定理:
    \begin{enumerate}
        \item 在一平面上取任一闭合区域,其面积为S,使它沿垂直于该区域的平面运动形成一个体积为V的立体,那么这个立体图形的体积就等于质心所经路程 r 乘以区域面积。表达式为V=S·r。
        \item 如果令某一长为L的曲线段,其长度为L,使它沿着垂直于它所在平面的方向扫过一个面积S,那么这个面积的大小就等于线段质心移动的距离r乘以线段的长度。表达式为S=L·r。
        \item 注意:是质心,而不是重心,因为除非重力场是均匀的,否则同一物质(系统)的质心与重心通常不在同一假想点上。
    \end{enumerate}
\end{frame}

\begin{frame}{质心系}
    原点取在质心上的平动参考系(坐标架不转动)

    质心系是“零动量系”,满足$\sum_im_i\vec{v}_i'=0$

    若$\vec{F}_{ex}=0$,,质心系为惯性系,否则为非惯性系(须考虑惯性力)

    质心系中,惯性力总功为 0,因此不论质心系是否为惯性系,都不需要考虑惯性力做功

    惯性力相对于质心总力矩为 0
\end{frame}

\begin{frame}{变质量物体的运动}
    \begin{equation}
        m\frac{d\vec{v}}{dt}+(\vec{v}-\vec{u})\frac{dm}{dt}=\vec{F}
    \end{equation}

    \begin{figure}[htpb]
        \begin{center}
            \includegraphics[width=0.95\linewidth]{pic/4.png}
        \end{center}
    \end{figure}
\end{frame}

\begin{frame}{变质量物体的运动}
    \begin{figure}[htpb]
        \begin{center}
            \includegraphics[width=0.95\linewidth]{pic/5.png}
        \end{center}
    \end{figure}
\end{frame}

\subsection{动能定理}

\begin{frame}
    功:$W=\int_{\vec{r}_1}^{\vec{r}_2}\vec{F}\cdot d\vec{r}$,标量,可正可负

    功是相对量:位移是相对量;同一力在不同参考系中的功不一样

    功率:$P=\frac{dW}{dt}$,标量,可正可负

    动能:$E=\frac{1}{2}mv^2$,标量,可正可负
\end{frame}

\begin{frame}{质点系动能定理}
    作用于质点系的所有外力所作之功与所有内力所作之功的总和,等于质点系动能的增量

    \begin{equation}
        \begin{cases}
            W_{ext}&=\sum_{i=1}^{n}W_{i,ext}\\
            W_{int}&=\sum_{i=1}^{n}W_{i,int}=\sum_{i,j=1,i<j}^{n}\int_{\vec{r}_{ij0}}^{\vec{r}_{ij}}\vec{f}_{ij}\cdot d\vec{r}_{ij}\\
            E_k&=\sum_{i=1}^{n}E_{ki}=\sum_{i=1}^{n}\frac{1}{2}mv_i^2\\
            \Delta E&=W_{ext}-W_{int}
        \end{cases}
    \end{equation}
\end{frame}

\begin{frame}
    动能、做功与参考系选取有关。

    动能定理只适用于惯性系。非惯性系中引入惯性力作功,它与真实力作功之和也等于质点在非惯性系中动能的增量。

    内力的总冲量虽然为零,但内力的总功一般不为零。

    摩擦力做功,当摩擦力为体系外力时,对体系可能做正功,也可能做负功 , 也可能不做功;动摩擦力总是消耗体系的机械能,是一种耗散力;而静摩擦力不同,它不消耗体系的机械。

    质点系动量定理是矢量式,而质点系动能定理是标量式。

    质点系动量定理与质点系动能定理是相互独立的。

    内力的作用不改变体系的总动量,但一般要改变体系的总动能。
\end{frame}

\begin{frame}{质心系中的动能}
    \begin{figure}[htpb]
        \begin{center}
            \includegraphics[width=0.95\linewidth]{pic/6.png}
        \end{center}
    \end{figure}
\end{frame}

\subsection{势能}

\begin{frame}{有心力}
    \begin{figure}[htpb]
        \begin{center}
            \includegraphics[width=0.95\linewidth]{pic/7.png}
        \end{center}
    \end{figure}
\end{frame}

\begin{frame}{保守力}
    \begin{figure}[htpb]
        \begin{center}
            \includegraphics[width=0.95\linewidth]{pic/8.png}
        \end{center}
    \end{figure}
\end{frame}

\begin{frame}{势能}
    \begin{figure}[htpb]
        \begin{center}
            \includegraphics[width=0.95\linewidth]{pic/9.png}
        \end{center}
    \end{figure}
\end{frame}

\begin{frame}{常见的势能}
    \begin{enumerate}
        \item 重力势能:以地面为势能零点,$V(z)=\int_{0}^{z}mg\cdot dz=mgz$
        \item 引力势能:质量各为$M$,$m$的两质点的引力势能,取$M$为原点,无穷远处为势能零点,$V(r)=\int_{\infty}^{r}G\frac{Mm}{r^2}=-G\frac{Mn}{r}$
        \item 弹性势能:以弹簧原长处的势能为零,$V(x)=\int_{0}^{x}kx\cdot dx=\frac{kx^2}{2}$
        \item 双原子分子势能(Morse势):$V(r)=D_e(e^{-2a(r-r_0)}-2e^{-a(r-r_0)})$
    \end{enumerate}
\end{frame}

\begin{frame}{势能曲线}
    \begin{figure}[htpb]
        \begin{center}
            \includegraphics[width=0.95\linewidth]{pic/10.png}
        \end{center}
    \end{figure}
\end{frame}

\begin{frame}{势能曲线}
    \begin{figure}[htpb]
        \begin{center}
            \includegraphics[width=0.95\linewidth]{pic/11.png}
        \end{center}
    \end{figure}
\end{frame}

\begin{frame}{功能原理和机械能守恒定律}
    \begin{figure}[htpb]
        \begin{center}
            \includegraphics[width=0.95\linewidth]{pic/12.png}
        \end{center}
    \end{figure}
\end{frame}

\begin{frame}{功能原理和机械能守恒定律}
    \begin{figure}[htpb]
        \begin{center}
            \includegraphics[width=0.95\linewidth]{pic/13.png}
        \end{center}
    \end{figure}
\end{frame}

\begin{frame}{功能原理和机械能守恒定律}
    \begin{figure}[htpb]
        \begin{center}
            \includegraphics[width=0.95\linewidth]{pic/14.png}
        \end{center}
    \end{figure}
\end{frame}

\begin{frame}{宇宙速度}
    第一宇宙速度:$7.9 km/s$

    第二宇宙速度:$11.2 km/s$

    第三宇宙速度:$16.7 km/s$

    详见Prof. Xu的Chapter 3 Lecture P65-P72
\end{frame}

\subsection{两体问题}

\begin{frame}
    考虑两个质点的孤立体系,不受外力

    在质心系中处理:质心的运动+两质点的相对运动
    
    质心:静止或匀速直线运动,质心系为惯性系

    折合质量:$\mu=\frac{m_1m_2}{m_1+m_2}$,$\vec{F}=\mu\ddot{\vec{r}}$

    体系能量:$E=\frac{1}{2}\mu\vec{v}^2+V(r)$
\end{frame}

\subsection{碰撞}

\begin{frame}
    碰撞:两物体互相接近、运动状态发生急剧变化的过程

    碰前:两物体均处于自由运动状态,没有相互作用

    碰后:分离,回到各自的自由运动状态

    碰撞过程:发生相互作用,涉及动量和能量交换的过程

    碰撞前后系统动量守恒,有些过程中的内力远大于外力,而且作用时间极短,外力的冲量可以忽略,也可以看成是碰撞
\end{frame}

\begin{frame}
    弹性碰撞:碰撞过程中没有机械能的损失

    非弹性碰撞:碰撞后物体的形变不能完全消失,这时机械能不守恒

    一维碰撞:碰撞前后两质点的速度都在同一条直线上,也称为正碰

    二维碰撞:碰撞前后质点的速度在同一个平面内

    三维碰撞:碰撞前后不在同一个平面内
    
    二维、三维碰撞也称为斜碰
\end{frame}

\begin{frame}{正碰}
    \begin{figure}[htpb]
        \begin{center}
            \includegraphics[width=0.95\linewidth]{pic/15.png}
        \end{center}
    \end{figure}
\end{frame}

\begin{frame}{正碰}
    \begin{figure}[htpb]
        \begin{center}
            \includegraphics[width=0.95\linewidth]{pic/16.png}
        \end{center}
    \end{figure}
\end{frame}

\begin{frame}{正碰}
    \begin{figure}[htpb]
        \begin{center}
            \includegraphics[width=0.95\linewidth]{pic/17.png}
        \end{center}
    \end{figure}
\end{frame}

\begin{frame}{正碰}
    完全非弹性碰撞:$\Delta E_k=-\frac{1}{2}\mu(u_1-u_2)^2$

    资用能:质心动能在对撞前后不变,$m_1$,$m_2$在质心系中的动能之和称为资用能$E_{kc0}=-\frac{1}{2}\mu(u_1-u_2)^2$
\end{frame}

\begin{frame}{正碰}
    一般非弹性碰撞
    \begin{figure}[htpb]
        \begin{center}
            \includegraphics[width=0.95\linewidth]{pic/18.png}
        \end{center}
    \end{figure}
\end{frame}

\begin{frame}{斜碰}
    \begin{figure}[htpb]
        \begin{center}
            \includegraphics[width=0.95\linewidth]{pic/19.png}
        \end{center}
    \end{figure}
\end{frame}

\begin{frame}{质心坐标系中讨论碰撞}
    \begin{figure}[htpb]
        \begin{center}
            \includegraphics[width=0.95\linewidth]{pic/20.png}
        \end{center}
    \end{figure}
\end{frame}

\begin{frame}{质心坐标系中讨论碰撞}
    \begin{figure}[htpb]
        \begin{center}
            \includegraphics[width=0.95\linewidth]{pic/21.png}
        \end{center}
    \end{figure}
\end{frame}

\section{作业习题讲解}

\begin{frame}
    \begin{center}
        {\Huge 作业习题讲解}
    \end{center}
\end{frame}

\section{Q\&A}

\begin{frame}
    \begin{center}
        {\Huge\calligra Q\&A}
    \end{center}
\end{frame}

\begin{frame}
    \begin{center}
        {\Huge\calligra Thanks!}
    \end{center}
\end{frame}

\end{document}