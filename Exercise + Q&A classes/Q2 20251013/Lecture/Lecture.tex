\documentclass[aspectratio=169]{beamer}
\usepackage[UTF8]{ctex} % Use default font set
\usepackage{hyperref}

% other packages
\usepackage{latexsym,amsmath,xcolor,multicol,booktabs,calligra}
\usepackage{siunitx} % For \SI command
\usepackage{graphicx,pstricks,listings,stackengine}

\author{Yu Shu \& Chihao Shi}
\title{力学A(PHYS1001A.04):第二次习题课}
\subtitle{Course NOT easy: The Survival Guide 1+1}
\institute{School of Physics, USTC}
\date{Oct.13, 2025}
\usepackage{USTC}

% defs
\def\cmd#1{\texttt{\color{red}\footnotesize $\backslash$#1}}
\def\env#1{\texttt{\color{blue}\footnotesize #1}}
\definecolor{deepblue}{rgb}{0,0,0.5}
\definecolor{deepred}{rgb}{0.6,0,0}
\definecolor{deepgreen}{rgb}{0,0.5,0}
\definecolor{halfgray}{gray}{0.55}

\lstset{
    basicstyle=\ttfamily\small,
    keywordstyle=\bfseries\color{deepblue},
    emphstyle=\ttfamily\color{deepred},    % Custom highlighting style
    stringstyle=\color{deepgreen},
    numbers=left,
    numberstyle=\small\color{halfgray},
    rulesepcolor=\color{red!20!green!20!blue!20},
    frame=shadowbox,
}


\begin{document}

\kaishu

\begin{frame}
    \titlepage
    \begin{figure}[htpb]
        \begin{center}
            \includegraphics[width=0.15\linewidth]{pic/ustc_logo_fig-eps-converted-to.pdf}
        \end{center}
    \end{figure}
\end{frame}

\begin{frame}
    \tableofcontents[sectionstyle=show,subsectionstyle=show/shaded/hide,subsubsectionstyle=show/shaded/hide]
\end{frame}

\section{内容回顾与补充拓展}

\subsection{自然坐标系与微分几何}

\begin{frame}{自然坐标系}
    在质点运动的轨迹线上任取一点$O$为自然坐标原点,以质点所在位置$P$点与$O$点间轨迹的长度$s$来确定质点的位置,则称$s$为质点的自然坐标

    为描写平面内质点的运动,我们还需规定两个依赖于质点位置的单位矢量:
    \begin{itemize}
        \item $\vec{e_\tau}$:切向单位矢量,沿曲线切向,且指向自然坐标增大方向
        \item $\vec{e_n}$:法向单位矢量,沿曲线法向,且指向曲线的凹侧
    \end{itemize}

    $\vec{e_\tau}$和$\vec{e_n}$不是常矢量:长度不变,但是方向取决于质点的位置

    只是轨道上的点才对应有坐标和单位矢量,轨道外的点不能用自然坐标系来描写
\end{frame}

\begin{frame}{运动分解}
    \begin{figure}[htbp]
        \centering
        \includegraphics[width=0.95\textwidth]{pic/1.png}
    \end{figure}
\end{frame}

\begin{frame}{运动分解}
    \begin{figure}[htbp]
        \centering
        \includegraphics[width=0.95\textwidth]{pic/2.png}
    \end{figure}
\end{frame}

\begin{frame}{运动分解}
    \begin{figure}[htbp]
        \centering
        \includegraphics[width=0.95\textwidth]{pic/3.png}
    \end{figure}
\end{frame}

\begin{frame}{运动分解}
    \begin{figure}[htbp]
        \centering
        \includegraphics[width=0.95\textwidth]{pic/4.png}
    \end{figure}
\end{frame}

\begin{frame}{知识拓展:Fernet Frame}
    单位切向量$\mathbf{T}$,单位法向量$\mathbf{N}$,单位副法向量$\mathbf{B}$,被称作弗勒内标架,他们的具体定义如下:
    \begin{itemize}
        \item $\mathbf{T}$是与曲线相切的单位向量,指向运动方向。
        \item $\mathbf{N}$是法线单位向量,是切向量$\mathbf{T}$对弧长参数的微分单位化得到的向量。
        \item $\mathbf{B}$是$\mathbf{T}$与$\mathbf{N}$的外积。
    \end{itemize}

    我们也有一组方程,称为Ferne-Serret Formulas:
    \begin{equation*}
        \begin{cases}
            \frac{d\mathbf{T}}{dt} = \kappa\mathbf{N} \\
            \frac{d\mathbf{N}}{dt} = -\kappa\mathbf{T} + \tau\mathbf{B} \\
            \frac{d\mathbf{B}}{dt} = -\tau\mathbf{N}
        \end{cases}
    \end{equation*}
\end{frame}

\begin{frame}{知识拓展:Fernet Frame}
    \begin{equation*}
        \begin{aligned}
            &s(t) = \int_{0}^{t}||\vec{r}'(\tau)||d\tau \\
            &\mathbf{T}=\frac{d\vec{r}}{ds}
            &\mathbf{N}=\frac{\frac{d\mathbf{T}}{ds}}{||\frac{d\mathbf{T}}{ds}||} \\
            &\mathbf{B}=\mathbf{T}\times\mathbf{N} \\
            &\text{由于}|\mathbf{T}|=1,\frac{d\mathbf{T}\cdot T}{ds}=2\mathbf{T}\cdot\mathbf{N}=0,\mathbf{T}\perp\mathbf{N} \\
            &\text{故}\begin{bmatrix}
            &\mathbf{T}' \\ \mathbf{N}' \\ \mathbf{B}'
            &\end{bmatrix}=\begin{bmatrix}
                0 & \kappa & 0 \\ -\kappa & 0 & \tau \\ 0 & -\tau & 0
            \end{bmatrix}\begin{bmatrix}
                \mathbf{T} \\ \mathbf{N} \\ \mathbf{B}
            \end{bmatrix}
        \end{aligned}
    \end{equation*}
\end{frame}

\begin{frame}{知识拓展:Fernet Frame}
    \begin{figure}[htbp]
        \centering
        \includegraphics[width=0.95\textwidth]{pic/5.png}
    \end{figure}
\end{frame}

\begin{frame}{知识拓展:Fernet Frame}
    \begin{figure}[htbp]
        \centering
        \includegraphics[width=0.95\textwidth]{pic/6.png}
    \end{figure}
\end{frame}

\begin{frame}{知识拓展:Fernet Frame}
    以圆周运动为例
    \begin{figure}[htbp]
        \centering
        \includegraphics[width=0.95\textwidth]{pic/7.png}
    \end{figure}
\end{frame}

\subsection{运动的相对性}

\begin{frame}{相对运动}
    \begin{figure}[htbp]
        \centering
        \includegraphics[width=0.95\textwidth]{pic/8.png}
    \end{figure}
\end{frame}

\begin{frame}{平动}
    \begin{figure}[htbp]
        \centering
        \includegraphics[width=0.95\textwidth]{pic/9.png}
    \end{figure}
\end{frame}

\begin{frame}{平动}
    \begin{figure}[htbp]
        \centering
        \includegraphics[width=0.95\textwidth]{pic/10.png}
    \end{figure}
\end{frame}

\subsection{运动学总结}

\begin{frame}
    \begin{figure}[htbp]
        \centering
        \includegraphics[width=0.95\textwidth]{pic/11.png}
    \end{figure}
\end{frame}

\begin{frame}
    \begin{figure}[htbp]
        \centering
        \includegraphics[width=0.95\textwidth]{pic/12.png}
    \end{figure}
\end{frame}

\subsection{牛顿运动定律}

\begin{frame}{牛顿之前}
    \begin{itemize}
        \item 亚里士多德
        \item 哥白尼、开普勒、伽利略
        \item 笛卡尔、惠更斯、胡克
    \end{itemize}
\end{frame}

\begin{frame}{牛顿第一定律(惯性定律)}
    任何物体都保持静止状态或匀速直线运动状态,直到其它物体的作用迫使它改变这种状态为止

    \begin{itemize}
        \item 第一定律是大量观察与实验事实的抽象与概括,不能用实验直接验证。不可能不受其他物体作用
        \item 第一定律定性提出了“力”和“惯性”两个重要概念
        \begin{itemize}
            \item 惯性(Inertial):运动物体自身具有保持其匀速运动或静止的属性,这种属性称为“惯性”
            \item 力是一个物体对另一物体的相互作用。是改变物体运动状态的原因,而不是维持运动状态的原因
        \end{itemize}
        \item 惯性定律的意义是在于断言:一定存在着这样的参考系,相对于它,所有不受外力作用的物体都保持自己的速度
        \item 第一定律具有公理性
    \end{itemize}
\end{frame}

\begin{frame}{惯性系}
    牛顿第一定律定律成立的参考系为惯性参考系(L.Lange, 1885)。

    在惯性系中,从同一点向三个不同方向(非共面)扔出的质点都会直线运动。

    常用的惯性参考系:
    \begin{itemize}
        \item 地面参考系:又称实验室参考系。由于地球的自转造成的的加速度为$3.4\times 10^{-2}m/s^2$,在精度不太高时,自转的加速度效应可以忽略,因此可将地球参考系可以看作惯性参考系
        \item 地心参考系:即地球─恒星参考系,以地心为原点,坐标轴指向恒星的惯性参考系。地球公转造成的加速度为$5.9\times 10^{-3}m/s^2$,常在发射人造地球卫星时采用
        \item 太阳系:即太阳─恒星参考系,以太阳中心为原点,坐标轴指向其它恒星的惯性参考系。太阳对银河系中心的转动加速度为$10^{-10}m/s^2$, 常在在研究行星等天体运动时采用
        \item 基本星表参考系(Catalogues of Fundamental Stars,FK系):是以相对于选定的大量恒星的平均静止位形作为基准的参考系
        \item 马赫原理
    \end{itemize}
\end{frame}

\begin{frame}{牛顿第二定律}
    运动的改变与所加的外力成正比,并发生在所加的力的那个直线方向上

    \begin{equation*}
        \begin{aligned}
            \vec{p} &= m\vec{v} \\
            \vec{F} &= \frac{d\vec{p}}{dt}=\frac{dm}{dt}\vec{v}+m\frac{d\vec{v}}{dt}
        \end{aligned}
    \end{equation*}

    若物体的质量不随时间变化,则$\vec{F}=m\vec{a}$。
\end{frame}

\begin{frame}{牛顿第二定律}
    第二定律可认为既是定义又是定律

    在相同的力𝑭作用下的两个物体,质量与加速度成反比。
    设这两个物体的质量分别为$m_1$, $m_2$,加速度分别为$a_1$, $a_2$,则有:$m_1a_1=m_2a_2\Rightarrow m_2=m_1\frac{a_1}{a_2}$。
    若取$m_1$为标准质量,由于$a_1$, $𝑎_2$都是可以测量的,那么$𝑚_2$可以完全确定。
    这种用惯性大小定义的质量称为惯性质量。

    \begin{itemize}
        \item 惯性质量(Inertial mass):描述物体运动改变难易程度的物理量
        \item 引力质量(Gravitational mass):描述物体在一个引力场里受引力的大小
    \end{itemize}

    从17世纪以来不断有实验证明,惯性质量和引力质量是等价的,这条原理在广义相对论中称为等效原理(弱等效原理, weak equivalence principle)。
    核心实验是Eötvös实验
\end{frame}

\begin{frame}{牛顿第二定律}
    在2019年5月20日之前,千克仍是国际单位制基本单位中唯一仍使用实物进行定义的单位,即被定义为国际千克原器(以铂铱合金铸造)的质量。

    物理学中诸多的物理量都和千克有关,例如牛顿(力学单位),瓦特(功率单位),焦耳(能量单位)等,因此,千克原器任何轻微改变都会引起其他物理量定义的混乱。

    现行千克定义为:$1kg = \frac{h}{6.62607015\times 10^{−34}}m^{-2}s$。
\end{frame}

\begin{frame}{牛顿第二定律}
    \begin{itemize}
        \item 第二定律适用的参考系是惯性系
        \item 定律中的外力是合外力$\vec{F}=\sum_{i=0}^{n}\vec{F_i}$
        \item 第二定律是瞬时关系式:当有力作用时,物体即产生加速度。 一旦力消失,加速度也随之消失,二者是同步的。牛顿第二定律描述的是一种超距作用(action at a distance)。
        \item 公式$\vec{F}=m\vec{a}$是矢量式
    \end{itemize}
\end{frame}

\begin{frame}{知识拓展:瓦特天平}
    \begin{figure}[htbp]
        \centering
        \includegraphics[width=0.95\textwidth]{pic/13.png}
    \end{figure}
\end{frame}

\begin{frame}{质量和力的定义}
    质量是绝对量:物体质量的度量值与物体的运动状态无关,在不同的参考系中质量$m$的度量值相同

    质量具有可加性:两个物体组合成大物体的质量等于两个物体质量的和

    在国际单位制中,力是导出量力的单位是牛顿($N$),$1N$的力使质量为$1kg$的物体产生$1m\cdot s^{-2}$的加速度
\end{frame}

\begin{frame}{“牛顿第零定律”}
    牛顿运动定律提出质量的概念,并建立在物体质量不随时间变化的基础上

    隐含“质量守恒”:物体的质量被认为是独立于其速度和任何时间与它的外力,总质量既不产生也不消灭,只是在物体相互作用的时候重新分配

    1773年,“现代化学之父”拉瓦锡(Antoine Lavoisier)提出质量守恒定律:参与化学反应的原子没有被消耗或产生,只是重新组合,因此原子的种类、数量和质量都守恒,从而保证了整个系统的质量守恒。

    \begin{figure}
        \centering
        \includegraphics[width=0.5\textwidth]{pic/14.png}
    \end{figure}
\end{frame}

\begin{frame}{牛顿第三定律}
    力与反作用力的描述:
    \begin{itemize}
        \item 作用力和反作用力总是成对地产生,并且同时存在、同时消失
        \item 作用力和反作用力是具有相同性质的力
        \item 有别于一对平衡力,作用力和反作用力分别作用于两个物体上,不能抵消
    \end{itemize}

    思考:重力与支持力是反作用力吗?离心力与向心力是作用力与反作用力吗?

    作用力与反作用力相等而反向,是以力的传递不需要时间,即传递速度无穷大为前提的。
    如果力以有限的速度传递,作用力和反作用力就不一定相等了。
\end{frame}

\subsection{单位制与量纲分析}

\begin{frame}
    详情可见第一次习题课讲义
\end{frame}

\subsection{力}

\begin{frame}{常见的力}
    力的分类:
    \begin{itemize}
        \item 接触力:两物体因接触而产生的相互作用力
        \item 非接触力:未接触时即存在的力
    \end{itemize}
\end{frame}

\begin{frame}{常见的力}
    弹性力(又称胡克力,elasticity):物体发生弹性变形后,内部产生欲恢复形变的力。
    
    常见的有:弹簧的弹力、绳索间的张力、压力、支持力等。

    \begin{equation*}
        \vec{F} = -k\Delta\vec{x}
    \end{equation*}
    
    $k$:弹性系数,由弹簧本身性质决定。负号表示弹性力与形变方向相反。
\end{frame}

\begin{frame}{常见的力}
    摩擦力(friction):物体与物体相互接触时,沿接触面两物体相互施以阻止相对滑动的作用力。

    分子力是产生摩擦的根本原因。

    静摩擦力: $0\leq f_s\leq \mu_s N$,$\mu_s$为静摩擦系数。
    方向:与物体相对滑动趋势的方向相反。
    大小:只能根据物体所处的运动状态,由平衡条件和牛顿定律来求解,与正压力$N$无关。

    滑动摩擦力:$f_k=\mu_k N$,$\mu_k$为动摩擦系数。
    方向:与物体相对运动的方向相反。

    湿摩擦(粘滞阻力, drag):固体与液体或者气体接触面发生相对运动时产生的阻力。
    低速下,常有$f=-bv$,速度较大情况下有$f=-cv^2$。
    一般情况下,粘性流体的运动由N-S方程描述。
    \begin{equation*}
        \begin{cases}
            \rho\frac{D\vec{u}}{Dt}=-\bigtriangledown p+\bigtriangledown\cdot{\mu[\bigtriangledown u+(\bigtriangledown\vec{u})^{T}-\frac{2}{3}(\bigtriangledown\cdot\vec{u})I]}+\rho\vec{a},\text{可压缩流体} \\
            \frac{D\vec{u}}{Dt} = \nu\bigtriangledown^2\vec{u}-\bigtriangledown(\frac{p}{\rho})+\vec{f},\text{不可压缩流体}
        \end{cases}
    \end{equation*}
\end{frame}

\begin{frame}{常见的力}
    万有引力:自然界任何两物体之间都存在着相互吸引力,叫做万有引力。
    两质点间万有引力大小与两质点的质量乘积成正比,与两质点间的距离平方成反比,力的方向沿着两质点的连线。
    \begin{equation*}
        F = \frac{Gm_1m_2}{r^2}
    \end{equation*}

    公式中的质量称为引力质量,它是物体与其他物体相互吸引性质的量度。

    重力:地球对表面物体的万有引力,$\vec{F}=m\vec{g}$。
\end{frame}

\begin{frame}{常见的力}
    库仑力:两个静止的点电荷之间的作用力的大小与它们电荷$q_1$,$q_2$的乘积成正比,与它们之间的距离$r$的平方成反比,方向沿着两点电荷的连线。
    如果电荷是异号的,则为吸引力,如果是同号的,则是排斥力。

    思考:万有引力和库仑力数学形式上的一致性,是否有更深层次的原因?

    洛伦兹力:带电质点在电磁场中所受力,$\vec{F}=q(\vec{E}+\vec{v}\times\vec{B})$。

    分子力:分子间相互作用的规律较复杂,很难用简单的数学公式来表示。
    一般在实验的基础上,采用简化模型处理问题,可近似地用半经验公式$F=\frac{\lambda}{r^s}-\frac{\mu}{r^t}$来描述,其中$s>t$。
\end{frame}

\begin{frame}{基本相互作用}
    \begin{table}[htbp]
        \centering
        \caption{基本相互作用}
        \begin{tabular}{|c|c|c|c|}
            \hline
            \textbf{力} & \textbf{相对强度} & \textbf{作用范围} & \textbf{描述性质} \\
            \hline
            强相互作用(Strong force) & 1 & 10$^{-15}$m & 结合原子核 \\
            \hline
            电磁力(Electromagnetic force) & 10$^{-2}$ & $\infty$ & 除引力外的宏观力 \\
            \hline
            弱相互作用(Weak force) & 10$^{-5}$ & 10$^{-15}$m & 原子核衰变 \\
            \hline
            引力(Gravitational force) & 10$^{-39}$ & $\infty$ & 组织宇宙 \\
            \hline
        \end{tabular}
    \end{table}
\end{frame}

\begin{frame}{基本相互作用}
    \begin{multicols}{2}
        基本相互作用的统一:
        \begin{itemize}
            \item 1967年,温伯格和萨拉姆提出电磁和弱相互作用的统一理论,并被实验所验证(1979年Nobel Prize)。
            \item 大统一理论(Grand Unified Theory):将强相互作用统一进来。
            \item 万有理论(Theory of Everything):统合四种基本相互作用,目前被认为最有可能成功的万有理论是弦理论(string theory)和圈量子引力论(loop quantum gravity)。
        \end{itemize}

        \begin{figure}[htbp]
            \centering
            \includegraphics[width=0.5\textwidth]{pic/15.png}
        \end{figure}
    \end{multicols}
\end{frame}

\subsection{牛顿运动定律的应用}

\begin{frame}
    \begin{figure}[htbp]
        \centering
        \includegraphics[width=0.95\textwidth]{pic/16.png}
    \end{figure}
\end{frame}

\begin{frame}
    \begin{figure}[htbp]
        \centering
        \includegraphics[width=0.95\textwidth]{pic/17.png}
    \end{figure}
\end{frame}

\section{作业习题讲解}

\begin{frame}
    \begin{center}
        {\Huge 作业习题讲解}
    \end{center}
\end{frame}

\section{Q\&A}

\begin{frame}
    \begin{center}
        {\Huge\calligra Q\&A}
    \end{center}
\end{frame}

\begin{frame}
    \begin{center}
        {\Huge\calligra Thanks!}
    \end{center}
\end{frame}

\end{document}