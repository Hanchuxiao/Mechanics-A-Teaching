\documentclass[UTF8,8pt,a4paper]{article}
\usepackage{ctex}
\usepackage{geometry}
\usepackage{amsmath}
\usepackage{hyperref}
\usepackage{graphicx}

\geometry{left=1.8cm,right=1.8cm,top=1.9cm,bottom=1.9cm}
\makeatletter
\def\@fnsymbol#1{\ensuremath{%
  \ifcase#1\or % 0
    \dagger\or % 1 - 改为 dagger
    *\or       % 2 - 原来的 dagger 变成 asterisk
    \ddagger\or % 3
    \S\or      % 4
    \P\or      % 5
    \|\or      % 6 (DOUBLE VERTICAL LINE)
    **\or      % 7
    \dagger\dagger\or % 8
    \ddagger\ddagger % 9
  \else
    \@ctrerr
  \fi
}}
\makeatother

\title{\huge\textbf{第一章习题解答}}
\author{TA:疏宇\thanks{School of Gifted Young, USTC, email:\url{shuyu2023@mail.ustc.edu.cn}}\quad师驰昊\thanks{School of Gifted Young, USTC, email:\url{1984019655@qq.com}}}
\date{October 5$^{\{\text{th}\}}$, 2025}

\begin{document}
\maketitle

\section{I 简答题}
简答题没有标准答案,以下仅提供思路分析.
\newline


\textbf{1 $|\frac{d\vec{v}}{dt}|=0$与$\frac{d|\vec{v}|}{dt}=0$各代表什么样的运动?两者有无区别?}

要理解$|\frac{d\vec{v}}{dt}|=0$与$|\frac{d|\vec{v}|}{dt}|=0$的物理意义及区别,需从矢量微积分和运动学概念出发分析.

对于$|\frac{d\vec{v}}{dt}|=0$,$\frac{d\vec{v}}{dt}$即为加速度矢量$\vec{a}$,故$|\frac{d\vec{v}}{dt}|=|\vec{a}|=0$.
这意味着加速度大小为零,则速度$\vec{v}$既不改变大小,也不改变方向,对应静止或匀速直线运动(速度的大小、方向均恒定).

对于$|\frac{d|\vec{v}|}{dt}|=0$,$|\vec{v}$是速率,因此$\frac{d|\vec{v}|}{dt}$是速率对时间的变化率.
该式要求速率恒定(不随时间变化).
速率恒定仅要求速度的大小不变,但速度的方向可以变化.
因此,该式对应所有速率不变的运动.

二者的区别源于对“速度变化”的约束范围不同:
\begin{itemize}
  \item $|\frac{d\vec{v}}{dt}|=0$要求速度的大小和方向均不变(加速度为零),因此只对应匀速直线运动.
  \item $|\frac{d|\vec{v}|}{dt}|=0$ 仅要求速率(速度大小)不变,方向可变,因此包含“匀速直线运动”和“匀速圆周运动”等更广的运动类型.
\end{itemize}

具体举例说明:
\begin{itemize}
  \item 匀速直线运动:同时满足两者(加速度为零 → 速度大小、方向均不变).
  \item 匀速圆周运动:仅满足$\frac{d|\vec{v}|}{dt}=0$(速率恒定),但不满足$|\frac{d\vec{v}}{dt}|=0$(存在向心加速度,加速度大小不为零).
\end{itemize}

综上,$|\frac{d\vec{v}}{dt}|=0$是“匀速直线运动”的充分必要条件;而$|\frac{d|\vec{v}|}{dt}|=0$是“速率不变运动”的充分必要条件,前者是后者的子集(匀速直线运动是速率不变运动的一种特殊情况).
\newline


\textbf{2 当物体的加速度恒定不变时,它的运动方向可否改变?为什么?}

当物体的加速度恒定不变时,它的运动方向可以改变.

加速度是速度的变化率($\vec{a}=\frac{d\vec{v}}{dt}$),反映速度(矢量,包含大小和方向)变化的快慢与方向.
加速度恒定时,其大小和方向均不变,但速度的变化由“加速度与速度的方向关系”决定.

当加速度与初速度 不在同一直线上时,物体做曲线运动,运动方向时刻变化.
例如平抛运动:物体初速度水平,加速度为竖直向下的重力加速度(大小、方向均恒定).
由于加速度与速度存在夹角,速度方向会从“水平”逐渐偏转向“倾斜向下”,运动轨迹为抛物线,方向持续改变.

当加速度与初速度在同一直线上时,物体做匀变速直线运动,运动方向也可能改变(当速度减为零后反向).
例如竖直上抛运动:物体初速度竖直向上,加速度为竖直向下的重力加速度(恒定).
上升阶段速度逐渐减小(方向仍向上);到最高点速度为零后,速度方向变为向下,运动方向从“向上”改为“向下”.

加速度恒定时,只要加速度与速度方向不完全相同且不为零,速度的方向就可能发生变化(曲线运动中方向持续偏转,或直线运动中方向反向).
因为加速度作为速度变化的“驱动”,会持续改变速度这个矢量的大小或方向,而运动方向是速度矢量的固有属性之一.
\newline


\textbf{3 分析下列说法的正确性:(1)作曲线运动的物体,必有切向加速度;(2)作曲线运动的物体,必有法向加速度;(3)具有加速度的物体,其速率必随时间改.}

\textbf{(1)作曲线运动的物体,必有切向加速度}

\textbf{错误}

切向加速度$a_\tau$由速度大小(速率)的变化引起,公式为$a_\tau=\frac{dv}{dt}$($v$为瞬时速率).
若物体做匀速率曲线运动(如匀速圆周运动),速率$v$不变,则$\frac{dv}{dt}=0$,此时切向加速度为零.
因此,曲线运动不一定存在切向加速度.

\textbf{(2)作曲线运动的物体,必有法向加速度}

\textbf{正确}

法向加速度$a_n$由速度方向的变化引起,公式为$a_n=\frac{v^2}{\rho}$($\rho$为曲率半径).
曲线运动中,速度方向时刻改变,因此法向加速度一定存在(即使速率不变,只要轨迹是曲线,$a_n\neq 0$).

\textbf{(3)具有加速度的物体,其速率必随时间改变}

\textbf{错误}

加速度是速度(矢量,包含大小和方向)的变化率.
若加速度只改变速度的方向而不改变大小(如匀速圆周运动),则物体的速率(速度的大小)保持不变.
因此,存在加速度时,速率不一定随时间改变.
\newline


\textbf{4 篮球运动员跑步投篮时,若瞄准篮投反而投不进,为什么?应如何投才能投准?}

跑步时,运动员(和球)具有水平方向的初速度(与跑步方向一致).
若直接“瞄准篮筐”投篮,相当于只考虑了垂直于篮筐的投篮速度,却忽略了球原有的水平初速度.

投出后,球的运动是“水平初速度+垂直投篮速度+重力加速度”的合成(类似斜抛运动).
此时:
\begin{itemize}
  \item 水平方向:球会保持跑步时的水平速度继续向前飞行;
  \item 垂直方向:球向篮筐运动,但因水平速度的存在,合运动轨迹会偏离原本“瞄准篮筐”的直线方向.
\end{itemize}

因此,球会因水平初速度“飞过”或“偏出”篮筐,而非准确落入.

要让跑步投篮的球准确入筐,核心是“让球相对于篮筐的运动轨迹准确指向篮筐”,需从以下方面调整:
\begin{itemize}
  \item 调整出手时机与力度:跑步过程中,需在提前量的位置出手(而非正对篮筐时再投).例如:向前跑动时,要在距离篮筐“稍前方”的位置投篮,让球的水平初速度与垂直投篮速度合成后,轨迹恰好指向篮筐.同时,出手力度需略小于静止投篮,以抵消水平速度的影响.
  \item 调整出手角度:跑步时,出手角度需比静止投篮更陡峭(即增加垂直分速度的比例).这样能削弱水平初速度对轨迹的干扰,让球更快“克服”水平速度,向篮筐方向汇聚.
  \item 利用“缓冲动作”抵消水平速度:投篮瞬间,通过手腕、手指的“缓冲”动作(如短暂持球再出手),让球与手短暂接触,削弱跑步带来的水平初速度,使球相对静止地脱离手掌,再以接近静止投篮的轨迹飞向篮筐.
\end{itemize}

跑步投篮的本质是“运动中的相对静止”,需通过出手时机、角度、力度的调整,让球在保持水平初速度的同时,合运动轨迹能准确指向篮筐,而非简单套用静止时的“直线瞄准”逻辑.

注:专业篮球训练中,跑步投篮还会结合步伐、身体平衡等细节,但物理原理的核心是“补偿运动状态下的初速度”,让球的相对运动轨迹准确.
\newline


\section{II 教材习题}
教材习题部分答案与教材给出答案有出入,以本习题解析为准.
\newline


\textbf{TB1.1 甲乙两列火车在同一水平直路上以相等的速率($30km/h$)相向而行.当它们相隔$60km$的时候,一只鸟以$60km/h$的恒定速度离开甲车头向乙车头飞去,当到达立即返回,如此来回往返不止.试求:}

\textbf{(1) 当两车头相遇时,鸟往返了多少次?}

两列火车与鸟均做匀速直线运动,鸟每次从一辆火车飞向另一辆火车再返回时,由于火车在持续靠近,鸟往返一次所需的时间会逐渐缩短,且这一过程会无限重复(类似“无穷级数”的思想,每次间隔时间趋近于0但次数无限).
因此,当两车头相遇时,鸟往返了无穷多次.

但是,实际上,考虑到火车间距离越来越近,在某一时刻起,鸟的长度已经不能忽略不计,鸟不能视为质点,此后终到某一时刻,鸟无法再完成完整的往返飞行.
故,当两车头相遇时,鸟往返了无穷多次,但实际只往返了有限次.

\textbf{(2) 鸟共飞行了多少时间及距离?}

鸟从飞离甲车到两车相遇的这段时间内始终在飞行,因此鸟飞行的总时间等于两列火车从初始相距60km到相遇所需的时间.

两列火车相向而行,相对速度为两者速度之和:$v_{rel}=30km/h+30km/h=60km/h$.

初始间距$s_0=60km$,则两车相遇所需时间即鸟的飞行时间:$t=\frac{s_0}{v_{rel}}=\frac{60km}{60km/h}=1h$.

鸟的飞行速度$v_{bird}=60km/h$,飞行距离为速度与时间的乘积:$s_{bird}=v_{bird}t=60km/h\times 1h=60km$.

故鸟共飞行了$1h$,飞行距离为$60km$.
\newline


\textbf{TB1.3 一物体做直线运动,它的位置由方程$x=10t^2+6$决定,其中$x$的单位为$m$,$t$的单位为$s$.试计算:}

\textbf{(1) 在$3.00~3.10s$、$3.00~3.01s$及$3.000~3.001s$间隔时间内的平均速度;}

$3.00s~3.10s$:
\begin{equation*}
  \bar{v}=\frac{x_2-x_1}{t_2-t_1}=\frac{(10\times3.10^2+6)m-(10\times3.00^2+6)m}{3.10s-3.00s}=61.0m/s
\end{equation*}

$3.00s~3.01s$:
\begin{equation*}
  \bar{v}=\frac{x_2-x_1}{t_2-t_1}=\frac{(10\times3.01^2+6)m-(10\times3.00^2+6)m}{3.01s-3.00s}=60.1m/s
\end{equation*}

$3.00s~3.001s$:
\begin{equation*}
  \bar{v}=\frac{x_2-x_1}{t_2-t_1}=\frac{(10\times3.001^2+6)m-(10\times3.00^2+6)m}{3.001s-3.00s}=60.01m/s
\end{equation*}

\textbf{(2) 在$t=3.00s$时的瞬时速度;}

\begin{equation*}
  \begin{aligned}
    v&=\lim_{\Delta t\to0}\frac{\Delta x}{\Delta t}=\lim_{\Delta t\to0}\frac{[10(t+\Delta t)^2+6]-[10t^2+6]}{\Delta t}=\lim_{\Delta t\to0}\frac{10(2t\Delta t+\Delta t^2)}{\Delta t}=\lim_{\Delta t\to0}\frac{20t\Delta t+10\Delta t^2}{\Delta t}=20t \\
     &=20\times 3m/s=60m/s
  \end{aligned}
\end{equation*}

\textbf{(3) 用微分方法求它的速度及加速度公式.}

\begin{equation*}
  \begin{cases}
    v=\dot{x}=\frac{d}{dt}(10t^2+6)=20t m/s \\
    a=\ddot{x}=\dot{v}=\frac{d}{dt}(20t)=20m/s^2
  \end{cases}
\end{equation*}
\newline


\textbf{TB1.25 在小山上安一靶子,由炮位所在处观测靶子的仰角为 $\alpha$,炮与靶子间的水平距离为 $L$,向目标射击时,炮身的仰角为 $\beta$.略去空气阻力,求能射中靶子的子弹的初速度 $v_0$.}

以炮口为坐标原点,水平方向为$x$轴,竖直方向为$y$轴.

已知炮位观测靶子的仰角为$\alpha$,炮与靶的水平距离为$L$,因此靶子的水平坐标$x=L$,竖直坐标$y=L\tan\alpha$.

炮身仰角为$\beta$,子弹初速度$v_0$可分解为水平和竖直分量,水平分速度$v_{0x}=v_0\cos\beta$,竖直分速度$v_{0y}=v_0\sin\beta$.

子弹的运动分为水平和竖直两个方向.
水平方向为匀速直线运动,$x=v_{0x}t$,代入$x=L$,得$t=\frac{L}{v_0\cos\beta}$.
竖直方向为竖直上抛运动,加速度为$-g$,$y=v_{0y}t-\frac{1}{2}gt^2$,代入$y=L\tan\alpha$和$t=\frac{L}{v_0\cos\beta}$,得:
\begin{equation*}
  L\tan\alpha=v_0\sin\beta\frac{L}{v_0\cos\beta}-\frac{1}{2}g(\frac{L}{v_0\cos\beta})^2
\end{equation*}

简化竖直方向的方程:
\begin{equation*}
  \begin{aligned}
    L\tan\alpha=L\tan\beta-\frac{gL^2}{2v_0^2\cos^2\beta} \\
    \Rightarrow \tan\alpha=\tan\beta-\frac{gL}{2v_0^2\cos^2\beta} \\
    \Rightarrow \tan\beta-\tan\alpha=\frac{gL}{2v_0^2\cos^2\beta}
  \end{aligned}
\end{equation*}

利用三角恒等式化简$\tan\beta-\tan\alpha$:
\begin{equation*}
  \tan\beta-\tan\alpha=\frac{\sin\beta}{\cos\beta}-\frac{\sin\alpha}{\cos\alpha}=\frac{\sin\beta\cos\alpha-\sin\alpha\cos\beta}{\cos\beta\cos\alpha}=\frac{\sin(\beta-\alpha)}{\cos\beta\cos\alpha}
\end{equation*}

代入上方程:
\begin{equation*}
  \tan\beta-\tan\alpha=\frac{\sin(\beta-\alpha)}{\cos\beta\cos\alpha}=\frac{gL}{2v_0^2\cos^2\beta}
\end{equation*}

解出$v_0^2$:
\begin{equation*}
  v_0^2=\frac{gL\cos\beta\cos\alpha}{2\cos^2\beta\sin(\beta-\alpha)}=\frac{gL\cos\alpha}{2\cos\beta\sin(\beta-\alpha)}
\end{equation*}

开平方得到初速度$v_0$:
\begin{equation*}
  v_0=\sqrt{\frac{gL\cos\alpha}{2\cos\beta\sin(\beta-\alpha)}}
\end{equation*}
\newline


\textbf{TB1.31 一物体从静止出发沿半径 $ R = 3.0\,\text{m} $ 的圆周运动,切向加速度 $ a_t = 3.0\,\text{m/s}^2 $.试问:}

\textbf{(1) 经过多长时间它的总加速度 $ \vec{a} $ 恰与半径成 $ 45^\circ $ 角?}

在圆周运动中,总加速度$\vec{a}$由切向加速度$a_\tau$和法向加速度$a_n$合成,且两者垂直.

当总加速度$\vec{a}$与半径成$45^\circ$角时,由于法向加速度$a_n$的方向与半径方向相反(法向加速度指向圆心,半径从圆心指向物体),此时$a_\tau$与$a_n$的大小关系满足$\tan 45^\circ=\frac{a_\tau}{a_n}$,即$a_\tau=a_n$.

已知切向加速度$a_\tau=3.0m/s^2$,则法向加速度$a_n$也应为$3.0m/s^2$.
法向加速度有$a_n=\frac{v^2}{R}$.
物体从静止出发(初速度$v_0=0$),由匀加速直线运动规律$v=v_0+a_\tau t$,则$v=a_\tau t$.

结合以上条件,易得$a_\tau=\frac{(a_\tau t)^2}{R}$,代入数值,得$t=1.0s$.

\textbf{(2) 在上述时间内物体所通过的路程 $ s $ 等于多少?}

物体从静止出发,切向加速度恒定,路程$s$(标量,表示轨迹长度)满足匀加速直线运动公式:
\begin{equation*}
  s=v_0t+\frac{1}{2}a_\tau t^2
\end{equation*}

代入数值,得$s=1.5m$.
\newline


\textbf{TB1.33 一杆以匀角速度 $\omega_0$ 绕其固定端 $O$ 且垂直于杆的轴转动.在 $t = 0$ 时,位于 $O$ 点的小球从相对于杆静止开始沿杆作加速度为 $a_0$ 的匀加速运动.求小珠在时刻 $t$ 的速度和加速度.}

由于小珠是在水平面上运动,采用极坐标系,原点在$O$点,同时规定径向单位矢量$\hat{r}$和切向单位矢量$\hat{\theta}$.

杆旋转角速度为$\omega_0$,故$\theta(t)=\omega_0 t$.

小珠沿杆运动,即沿径向运动.
由于它从$O$点出发,初速为零,加速度为$a_0$(相对于杆),则其相对于杆的径向位置为$r(t)=\frac{1}{2}a_0 t^2$.

在极坐标中,一个质点的速度公式为:
\begin{equation*}
  \vec{v}=\dot{r}\hat{r}+r\dot{\theta}\hat{\theta}
\end{equation*}

其中,$\dot{r}=\frac{dr}{dt}=a_0 t$,$\dot{\theta}=\frac{d\theta}{dt}=\omega_0$.

代入$r(t)=\frac{1}{2}a_0 t^2$,得:
\begin{equation*}
  \vec{v}=(a_0 t)\hat{r}+(\frac{1}{2}a_0 t^2)\omega_0\hat{\theta}=a_0 t\hat{r}+\frac{1}{2}a_0\omega_0 t^2\hat{\theta}
\end{equation*}

在极坐标中,加速度有:
\begin{equation*}
    \vec{a}=\dot{\vec{v}}=\frac{d}{dt}(\dot{r}\hat{r}+r\dot{\theta}\hat{\theta})=(\ddot{r}-r\dot{\theta}^2)\hat{r}+(r\ddot{\theta}+2\dot{r}\dot{\theta})\hat{\theta}
\end{equation*}

逐项计算,$r=\frac{1}{2}a_0 t$,$\dot{r}=a_0 t$,$\ddot{r}=a_0$,$\dot{\theta}=\omega_0$,$\ddot{\theta}=0$.

则径向分量:
\begin{equation*}
  a_r=\ddot{r}-r\dot{\theta}^2=a_0-(\frac{1}{2}a_0 t^2)\omega_0^2
\end{equation*}

切向分量:
\begin{equation*}
  a_\theta=r\ddot{\theta}+2\dot{r}\dot{\theta}=2a_0 \omega_0 t
\end{equation*}

因此,加速度矢量为:
\begin{equation*}
  \vec{a}(t)=a_r\hat{r}+a_\theta\hat{\theta}=(a_0-\frac{1}{2}a_0 \omega_0^2 t^2)\hat{r}+2a_0 \omega_0 t\hat{\theta}
\end{equation*}

注:切向加速度为$2a_0 \omega_0 t$,是科里奥利加速度(Coriolis acceleration)的表现.

故,答案为:
\begin{equation*}
  \begin{cases}
    \vec{v}(t)=a_0 t\hat{r}+\frac{1}{2}a_0\omega_0 t^2\hat{\theta} \\
    \vec{a}(t)=(a_0-\frac{1}{2}a_0 \omega_0^2 t^2)\hat{r}+2a_0 \omega_0 t\hat{\theta}
  \end{cases}
\end{equation*}
\newline


\section{III 补充习题}
\textbf{1 设有矢量$\vec{A}=3\vec{\imath}+4\vec{\jmath}-4\vec{k}$,($\vec{\imath}$,$\vec{\jmath}$,$\vec{k}$分别为$x$,$y$,$z$方向的单位矢量)}

\textbf{(a) 在$x-y$平面内找到一个单位向量$\hat{\vec{B}}$,使其垂直于向量$\vec{A}$;}

在$x-y$平面内的向量$\hat{\vec{B}}$满足$z$分量为零,故$\hat{\vec{B}}=a\vec{\imath}+b\vec{\jmath}+0\vec{k}$.

$\hat{\vec{B}}$为单位向量,模长为1,故:
\begin{equation*}
  \sqrt{a^2+b^2}=1
\end{equation*}

$\hat{\vec{B}}\bot\vec{A}$,故:
\begin{equation*}
  \vec{A}\cdot\hat{\vec{B}}=(3\vec{\imath}+4\vec{\jmath}-4\vec{k})\cdot(a\vec{\imath}+b\vec{\jmath}+0\vec{k})=3a+4b=0
\end{equation*}

解得$b=-\frac{3}{4}a$,代入模长方程,得:
\begin{equation*}
  \begin{cases}
    a=\frac{4}{5} \\ b=-\frac{3}{5}
  \end{cases}
\end{equation*}

或:
\begin{equation*}
  \begin{cases}
    a=-\frac{4}{5} \\ b=\frac{3}{5}
  \end{cases}
\end{equation*}

则$\hat{\vec{B}}=-\frac{4}{5}\vec{\imath}+\frac{3}{5}\vec{\jmath}$或$\hat{\vec{B}}=\frac{4}{5}\vec{\imath}-\frac{3}{5}\vec{\jmath}$.

\textbf{(b) 找到一个单位向量$\hat{\vec{C}}$,使得$\hat{\vec{C}}\bot\vec{A}$,$\hat{\vec{C}}\bot\hat{\vec{B}}$;}

方法一:设$\hat{\vec{C}}=a\vec{\imath}+b\vec{\jmath}+c\vec{k}$,由垂直条件和模长条件解得.(具体过程类似上问,略)

方法二:考虑外积的性质,$\vec{A}\times\hat{\vec{B}}$垂直于$\vec{A}$和$\hat{\vec{B}}$,且不为零矢量,因此可取单位化后的外积作为$\hat{\vec{C}}$.

取$\hat{\vec{B}}=\frac{4}{5}\vec{\imath}-\frac{3}{5}\vec{\jmath}$,计算外积:
\begin{equation*}
  \vec{A}\times\hat{\vec{B}}= \begin{vmatrix}
    \vec{\imath} & \vec{\jmath} & \vec{k} \\
    3 & 4 & -4 \\
    4/5 & -3/5 & 0
  \end{vmatrix} = -\frac{12}{5}\vec{\imath}-\frac{16}{5}\vec{\jmath}-\frac{25}{5}\vec{k}
\end{equation*}

归一化,得:
\begin{equation*}
  \hat{\vec{C}}=\frac{-\frac{12}{5}\vec{\imath}-\frac{16}{5}\vec{\jmath}-\frac{25}{5}\vec{k}}{\sqrt{\frac{12}{5}^2+\frac{16}{5}^2+\frac{25}{5}^2}}=-\frac{12}{\sqrt{1025}}\vec{\imath}-\frac{16}{\sqrt{1025}}\vec{\jmath}-\frac{25}{\sqrt{1025}}\vec{k}
\end{equation*}

同样地,我们也可以取$\hat{\vec{B}}=-\frac{4}{5}\vec{\imath}+\frac{3}{5}\vec{\jmath}$,计算外积,得到另一个方向的$\hat{\vec{C}}=\frac{12}{\sqrt{1025}}\vec{\imath}+\frac{16}{\sqrt{1025}}\vec{\jmath}+\frac{25}{\sqrt{1025}}\vec{k}$.

\textbf{(c) 证明$\vec{A}$垂直于$\hat{\vec{B}}$和$\hat{\vec{C}}$所在的平面.}

要证明一个向量垂直于一个平面,只需证明它垂直于该平面内的两个不共线向量.

由于$\hat{\vec{B}}\bot\hat{\vec{C}}$,且二者模均不为零,则二者必不共线.
则矢量垂直于二者张成的平面等价于矢量同时垂直于$\hat{\vec{B}}$和$\hat{\vec{C}}$.

$\vec{A}\bot\hat{\vec{B}}$为 (a) 的题设条件,故无需证明.

$\vec{A}\bot\hat{\vec{C}}$也为 (b) 的题设条件,故无需证明.
我们也可以简单验证一下:
\begin{equation*}
  \vec{A}\cdot\hat{\vec{C}}=(3\vec{\imath}+4\vec{\jmath}-4\vec{k})\cdot(-\frac{12}{\sqrt{1025}}\vec{\imath}-\frac{16}{\sqrt{1025}}\vec{\jmath}-\frac{25}{\sqrt{1025}}\vec{k})=0
\end{equation*}

由$\vec{A}\bot\hat{\vec{B}}$且$\vec{A}\bot\hat{\vec{C}}$,可知$\vec{A}$垂直于$\hat{\vec{B}}$和$\hat{\vec{C}}$所在的平面. (Q.E.D.)
\newline


\textbf{2 设$\vec{a}=(1,-2,1)$,$\vec{b}=(1,-1,3)$,$\vec{c}=(2,5,-3)$,求$\vec{a}\times\vec{b}$,$(\vec{a}\times\vec{b})\cdot\vec{c}$,$\vec{a}\times(\vec{b}\times\vec{c})$.}

首先计算$\vec{a}\times\vec{b}$:
\begin{equation*}
  \vec{a}\times\vec{b}= \begin{vmatrix}
    \vec{\imath} & \vec{\jmath} & \vec{k} \\
    1 & -2 & 1 \\
    1 & -1 & 3
  \end{vmatrix} = (-5,-2,1)
\end{equation*}

对于$\vec{a}\times\vec{b}\cdot\vec{c}$:
\begin{equation*}
  \vec{a}\times\vec{b}\cdot\vec{c}=(-5,-2,1)\cdot(2,5,-3)=-10-10-3=-23
\end{equation*}

对于$\vec{a}\times(\vec{b}\times\vec{c})$,我们可以直接计算:
\begin{equation*}
  \vec{a}\times(\vec{b}\times\vec{c})= (1,-2,1)\times\begin{vmatrix}
    \vec{\imath} & \vec{\jmath} & \vec{k} \\
    1 & -1 & 3 \\
    2 & 5 & -3
  \end{vmatrix} = (1,-2,1)\times(-12,9,7)=\begin{vmatrix}
    \vec{\imath} & \vec{\jmath} & \vec{k} \\
    1 & -2 & 1 \\
    -12 & 9 & 7
  \end{vmatrix} = (-23,-19,-15)
\end{equation*}

很多时候,计算外积需要利用行列式,行列式的计算成本是比较大的,我们可以利用矢量三重积公式简化计算量:
\begin{equation*}
  \begin{aligned}
    \vec{a}\times(\vec{b}\times\vec{c}) &= (\vec{a}\cdot\vec{c})\vec{b}-(\vec{a}\cdot\vec{b})\vec{c} \\
                                        &= ((1,-2,1)\cdot(2,5,-3))(1,-1,3) - ((1,-2,1)\cdot(1,-1,3))(2,5,-3) \\
                                        &= (-11)(1,-1,3) - (6)(2,5,-3) = (-23,-19,-15)
  \end{aligned}
\end{equation*}
\newline


\textbf{3 求下列函数的导函数:}

\textbf{(a) $y=8x^3+x+7$;}

\begin{equation*}
  y'=(8x^3)'+x'+(7)'=24x^2+1
\end{equation*}

\textbf{(b) $y=(x+1)(x-1)\tan x$;}

\begin{equation*}
  \begin{aligned}
    y'&=(x+1)'(x-1)\tan x+(x+1)(x-1)'\tan x+(x+1)(x-1)\tan'x \\
      &=(x-1)\tan x+(x+1)\tan x+(x+1)(x-1)\frac{1}{\cos^2 x} \\
      &=2x\tan x+\frac{(x+1)(x-1)}{\cos^2 x}
  \end{aligned}
\end{equation*}

\textbf{(c) $y=\frac{9x+x^2}{5x+6}$;}

\begin{equation*}
  \begin{aligned}
    y'&=\frac{(9x+x^2)'(5x+6)-(9x+x^2)(5x+6)'}{(5x+6)^2}=\frac{(9+2x)(5x+6)-(9x+x^2)5}{(5x+6)^2} \\
      &=\frac{5x^2+12x+54}{25x^2+60x+36}
  \end{aligned}
\end{equation*}

\textbf{(d) $y=x\cos x+\frac{\sin x}{x}$.}

\begin{equation*}
  \begin{aligned}
    y'&=(x\cos x)'+(\frac{\sin x}{x})'=x'\cos x+x\cos'x+\frac{\sin'x\cdot x-\sin x\cdot x'}{x^2} \\
      &=\cos x-x\sin x-\frac{x\cos x-\sin x}{x^2}
  \end{aligned}
\end{equation*}
\newline


\textbf{4 求不定积分:}

\textbf{(a) $\int (3x^3+\sin x+\frac{5}{x})dx$;}

\begin{equation*}
  \int (3x^3+\sin x+\frac{5}{x})dx=\int 3x^3dx+\int\sin xdx+\int\frac{5}{x}dx=\frac{3}{4}x^4 - \cos x + 5\ln|x| + C 
\end{equation*}

\textbf{(b) $\int \sqrt{a^2-x^2}dx(a>0)$.}

这是一个标准的三角代换积分.
令$x=a\sin\theta$,$-\frac{\pi}{2}\leq\theta\leq\frac{\pi}{2}$,则$dx=a\cos\theta d\theta$.

则:
\begin{equation*}
  \sqrt{a^2-x^2}=\sqrt{a^2-a^2\sin^2\theta}=\sqrt{a^2(1-\sin^2\theta)}=\sqrt{a^2\cos^2\theta}=a|\cos\theta|
\end{equation*}

由于$-\frac{\pi}{2}\leq\theta\leq\frac{\pi}{2}$范围内$\cos\theta\geq 0$,故:
\begin{equation*}
  \sqrt{a^2-x^2}=a\cos\theta
\end{equation*}

代入积分:
\begin{equation*}
  \int\sqrt{a^2-x^2}dx=\int(a\cos\theta)(a\cos\theta d\theta)=a^2\int\cos^2\theta d\theta
\end{equation*}

利用三角恒等式$\cos^2\theta=\frac{1+\cos 2\theta}{2}$,所以:
\begin{equation*}
  a^2\int\cos^2\theta d\theta=a^2\int\frac{1+\cos 2\theta}{2}d\theta=\frac{a^2}{2}\int(1+\cos 2\theta)d\theta=\frac{a^2}{2}(\theta+\frac{1}{2}\sin 2\theta)+C
\end{equation*}

注意到$\sin 2\theta=2\sin\theta\cos\theta$,故:
\begin{equation*}
  \frac{a^2}{2}(\theta+\frac{1}{2}\sin 2\theta)+C=\frac{a^2}{2}(\theta+\sin\theta\cos\theta)+C
\end{equation*}

将结果还原到$x$,$\theta=\arcsin\frac{x}{a}$,得到:
\begin{equation*}
  \begin{aligned}
    \sin\theta &= \frac{x}{a} \\
    \cos\theta &= \sqrt{1-\sin^2\theta}=\sqrt{1-\frac{x^2}{a^2}}=\frac{\sqrt{a^2-x^2}}{a} \\
    \sin\theta\cos\theta &= \frac{x\sqrt{a^2-x^2}}{a^2}
  \end{aligned}
\end{equation*}

代入,得:
\begin{equation*}
  \frac{a^2}{2}(\arcsin\frac{x}{a}+\frac{x\sqrt{a^2-x^2}}{a^2})+C=\frac{a^2}{2}\arcsin\frac{x}{a}+\frac{1}{2}x\sqrt{a^2-x^2}+C
\end{equation*}

最终答案为:
\begin{equation*}
  \int\sqrt{a^2-x^2}dx=\frac{a^2}{2}\arcsin\frac{x}{a}+\frac{1}{2}x\sqrt{a^2-x^2}+C
\end{equation*}
\newline


\textbf{5 验证函数$y=C_1\cos\omega t+C_2\sin\omega t$(其中$C_1$,$C_2$为任意函数)是微分方程$y''+\omega^2y=0$的解.}

要验证函数 $y = C_1 \cos \omega t + C_2 \sin \omega t$(其中 $C_1, C_2$ 是任意常数)是微分方程 $y'' + \omega^2 y = 0$ 的解,需将函数代入微分方程并验证等式成立.

对 $y = C_1 \cos \omega t + C_2 \sin \omega t$ 关于 $t$ 求导:
\begin{equation*}
  y' = \frac{d}{dt}(C_1 \cos \omega t) + \frac{d}{dt}(C_2 \sin \omega t) = -C_1 \omega \sin \omega t + C_2 \omega \cos \omega t
\end{equation*}

对 $y' = -C_1 \omega \sin \omega t + C_2 \omega \cos \omega t$ 再次对 $t$ 求导:
\begin{equation*}
  y'' = \frac{d}{dt}(-C_1 \omega \sin \omega t) + \frac{d}{dt}(C_2 \omega \cos \omega t) = -C_1 \omega \cdot \omega \cos \omega t + C_2 \omega \cdot (-\omega \sin \omega t)
\end{equation*}

简化得:
\begin{equation*}
  y'' = -\omega^2 C_1 \cos \omega t - \omega^2 C_2 \sin \omega t
\end{equation*}

微分方程为 $y'' + \omega^2 y = 0$,将 $y''$ 和 $y$ 代入方程:
\begin{equation*}
  \begin{aligned}
  y'' + \omega^2 y &= ( -\omega^2 C_1 \cos \omega t - \omega^2 C_2 \sin \omega t ) + \omega^2 ( C_1 \cos \omega t + C_2 \sin \omega t ) \\
                   &= -\omega^2 C_1 \cos \omega t - \omega^2 C_2 \sin \omega t + \omega^2 C_1 \cos \omega t + \omega^2 C_2 \sin \omega t \\
                   &= 0
  \end{aligned}
\end{equation*}

由于 $y'' + \omega^2 y = 0$ 恒成立,因此函数 $y = C_1 \cos \omega t + C_2 \sin \omega t$ 是微分方程 $y'' + \omega^2 y = 0$ 的解. (Q.E.D.)
\newline


\textbf{6 为了让电梯在上升、下降过程中平稳运行,将按照如下的加速度从静止启动,}

\textbf{$a(t)=(a_m/2)(1-\cos(2\pi t/T)),0\le t\le T$ }

\textbf{$a(t)=-(a_m/2)(1+\cos(2\pi t/T)),T\le t\le 2T$ }

\textbf{其中$a_m$是最大的加速度,$2T$是运行一趟(静止-运行-静止)所需的总时间.}

\textbf{(a) 电梯运行的最大速度是多少?}

速度是加速度的积分:
\begin{equation*}
  v(t) = \int_0^t a(\tau)\, d\tau
\end{equation*}

由于加速度在 $ [0, T] $ 内为正,电梯加速;在 $ [T, 2T] $ 内为负,电梯减速.
因此,最大速度出现在 $ t = T $ 时刻.

计算:
\begin{equation*}
  v(T) = \int_0^T a(t)\, dt = \int_0^T \frac{a_m}{2} \left(1 - \cos\left(\frac{2\pi t}{T}\right)\right)\, dt
\end{equation*}

拆分积分:
\begin{equation*}
  v(T) = \frac{a_m}{2} \int_0^T 1\, dt - \frac{a_m}{2} \int_0^T \cos\left(\frac{2\pi t}{T}\right)\, dt
\end{equation*}

计算各项:
\begin{equation*}
  \begin{aligned}
    \int_0^T 1\, dt = T \\
    \int_0^T \cos\left(\frac{2\pi t}{T}\right)\, dt = \left[ \frac{T}{2\pi} \sin\left(\frac{2\pi t}{T}\right) \right]_0^T = \frac{T}{2\pi} (\sin(2\pi) - \sin(0)) = 0
  \end{aligned}
\end{equation*}

因此:
\begin{equation*}
  v(T) = \frac{a_m}{2} \cdot T - 0 = \frac{a_m T}{2}
\end{equation*}

因此最大速度为$v_{\max} = \frac{a_m T}{2}$

\textbf{(b) 在电梯启动的很短时间内$t\ll T$,写出其速度的近似表达式;}

方法一:先精确积分,再泰勒展开.

先精确积分得到速度的表达式:
\begin{equation*}
  \begin{aligned}
    v(t) &= \int_0^t \frac{a_m}{2} \left(1 - \cos\left(\frac{2\pi \tau}{T}\right)\right) d\tau \\
         &= \frac{a_m}{2} \left[ \int_0^t 1\,d\tau - \int_0^t \cos\left(\frac{2\pi \tau}{T}\right) d\tau \right] \\
         &= \frac{a_m}{2} \left[ t - \frac{T}{2\pi} \sin\left(\frac{2\pi t}{T}\right) \right]
  \end{aligned}
\end{equation*}

令 $x = \dfrac{2\pi t}{T}$,当 $t \ll T$ 时,$x \ll 1$.  
对 $\sin x$ 做泰勒展开至三阶项:
\begin{equation*}
  \sin x = x - \frac{x^3}{6} + \mathcal{O}(x^5)
\end{equation*}

代入得:
\begin{equation*}
  \frac{T}{2\pi} \sin\left(\frac{2\pi t}{T}\right) = \frac{T}{2\pi} \left( \frac{2\pi t}{T} - \frac{1}{6} \left(\frac{2\pi t}{T}\right)^3 + \cdots \right) = t - \frac{2\pi^2 t^3}{3T^2} + \mathcal{O}(t^5)
\end{equation*}

代入速度表达式
\begin{equation*}
  \begin{aligned}
    v(t) &= \frac{a_m}{2} \left[ t - \left( t - \frac{2\pi^2 t^3}{3T^2} + \cdots \right) \right] \\
         &= \frac{a_m}{2} \cdot \frac{2\pi^2 t^3}{3T^2} + \mathcal{O}(t^5) \\
         &= \frac{\pi^2 a_m}{3T^2} t^3 + \mathcal{O}(t^5)
  \end{aligned}
\end{equation*}

则在$t\ll T$ 时,速度表达式为:
\begin{equation*}
  v(t) \approx \frac{\pi^2 a_m}{3T^2} t^3
\end{equation*}

方法二:先展开加速度,再积分.

利用 $\cos x = 1 - \dfrac{x^2}{2} + \dfrac{x^4}{24} - \cdots$,得:
\begin{equation*}
  1 - \cos x = \frac{x^2}{2} - \frac{x^4}{24} + \cdots
\end{equation*}

令 $x = \dfrac{2\pi t}{T}$,则:
\begin{equation*}
  \begin{aligned}
    a(t) &= \frac{a_m}{2} \left(1 - \cos\left(\frac{2\pi t}{T}\right)\right) \\
         &= \frac{a_m}{2} \left( \frac{1}{2} \left(\frac{2\pi t}{T}\right)^2 + \mathcal{O}(t^4) \right) \\
         &= \frac{a_m}{2} \cdot \frac{2\pi^2 t^2}{T^2} + \mathcal{O}(t^4) \\
         &= \frac{\pi^2 a_m}{T^2} t^2 + \mathcal{O}(t^4)
  \end{aligned}
\end{equation*}

积分求速度,得:
\begin{equation*}
  \begin{aligned}
    v(t) &= \int_0^t a(\tau)\,d\tau \\
         &= \int_0^t \left( \frac{\pi^2 a_m}{T^2} \tau^2 + \mathcal{O}(\tau^4) \right) d\tau \\
         &= \frac{\pi^2 a_m}{T^2} \cdot \frac{t^3}{3} + \mathcal{O}(t^5) \\
         &= \frac{\pi^2 a_m}{3T^2} t^3 + \mathcal{O}(t^5)
  \end{aligned}
\end{equation*}

则在$t\ll T$ 时,速度表达式为:
\begin{equation*}
  v(t) \approx \frac{\pi^2 a_m}{3T^2} t^3
\end{equation*}

事实上,从物理意义的角度分析,在 $t = 0$ 时,$a(0) = \frac{a_m}{2}(1 - \cos 0) = 0$,即电梯从静止开始平滑启动,无冲击.
加速度在初始时刻的变化率不为零,其主导行为为 $a(t) \propto t^2$(因为 $1 - \cos x \sim x^2/2$).
由于 $v(t) = \int_0^t a(\tau)\,d\tau$,若 $a(\tau) \sim \tau^2$,则 $v(t) \sim t^3$.
因此,速度在启动初期必然与 $t^3$ 成正比,比例系数由 $a_m$ 和 $T$ 决定.
通过上述两种数学方法可严格确定该系数为 $\dfrac{\pi^2 a_m}{3T^2}$.

\textbf{(c) 求电梯运行一趟距离为$D$所需要的时间.}

运行一趟的距离 $ D $ 等于速度对时间的积分:
\begin{equation*}
  D = \int_0^{2T} v(t)\, dt
\end{equation*}

由对称性:前半段 $ [0,T] $ 加速,后半段 $ [T,2T] $ 减速,且加速度对称,故速度曲线关于 $ t = T $ 对称.

因此:
\begin{equation*}
  D = 2 \int_0^T v(t)\, dt
\end{equation*}

而:
\begin{equation*}
  v(t) = \int_0^t a(\tau)\, d\tau = \frac{a_m}{2} \left( t - \frac{T}{2\pi} \sin\left(\frac{2\pi t}{T}\right) \right)
\end{equation*}

所以:
\begin{equation*}
  \int_0^T v(t)\, dt = \int_0^T \frac{a_m}{2} \left( t - \frac{T}{2\pi} \sin\left(\frac{2\pi t}{T}\right) \right)\, dt
= \frac{a_m}{2} \left[ \int_0^T t\, dt - \frac{T}{2\pi} \int_0^T \sin\left(\frac{2\pi t}{T}\right)\, dt \right]
\end{equation*}

计算:
\begin{equation*}
  \begin{aligned}
    \int_0^T t\, dt = \frac{1}{2} T^2 \\
    \int_0^T \sin\left(\frac{2\pi t}{T}\right)\, dt = \left[ -\frac{T}{2\pi} \cos\left(\frac{2\pi t}{T}\right) \right]_0^T = -\frac{T}{2\pi} (\cos(2\pi) - \cos(0)) = -\frac{T}{2\pi} (1 - 1) = 0
  \end{aligned}
\end{equation*}

因此:
\begin{equation*}
  \int_0^T v(t)\, dt = \frac{a_m}{2} \cdot \frac{1}{2} T^2 = \frac{a_m T^2}{4}
\end{equation*}

于是:
\begin{equation*}
  D = 2 \cdot \frac{a_m T^2}{4} = \frac{a_m T^2}{2}
\end{equation*}

解出 $ T $:
\begin{equation*}
  T^2 = \frac{2D}{a_m} \Rightarrow T = \sqrt{\frac{2D}{a_m}}
\end{equation*}

但注意:总时间为 $ 2T $,所以:
\begin{equation*}
  t_{\text{total}} = 2T = 2 \sqrt{\frac{2D}{a_m}} = 2\sqrt{2} \sqrt{\frac{D}{a_m}}
\end{equation*}
\newline


\textbf{7 如图1所示,一个运动员站在一倾角为$\phi$的山坡顶上往坡面扔石头,为使得石头扔出最远的距离,应该从水平面以多大的角度向上扔出石头?}
\begin{figure}[htbp]
  \centering
  \includegraphics[width=0.4\textwidth]{Fig/1.png}
  \caption{图1}
\end{figure}

设起点为原点 $O$,水平向右为 $x$ 轴,竖直向上为 $y$ 轴.
山坡向下倾斜,其表面满足$y = -x \tan\phi$.

石块的抛体运动方程为:
\begin{equation*}
  \begin{cases}
    x(t) = v_0 \cos\theta \cdot t, \\
    y(t) = v_0 \sin\theta \cdot t - \dfrac{1}{2} g t^2.
  \end{cases}
\end{equation*}

落地时满足轨迹与坡面相交,即:
\begin{equation*}
  v_0 \sin\theta \cdot t - \frac{1}{2} g t^2 = - (v_0 \cos\theta \cdot t) \tan\phi.
\end{equation*}

两边除以 $t \neq 0$ 得:
\begin{equation*}
  v_0 \sin\theta - \frac{1}{2} g t = - v_0 \cos\theta \tan\phi,
\end{equation*}

解得飞行时间$t = \frac{2v_0}{g} \left( \sin\theta + \cos\theta \tan\phi \right)$.

代入 $x(t)$ 得:
\begin{equation*}
  x = v_0 \cos\theta \cdot t = \frac{2v_0^2}{g} \cos\theta \left( \sin\theta + \cos\theta \tan\phi \right).
\end{equation*}

整理为:
\begin{equation*}
  x(\theta) = \frac{2v_0^2}{g} \left( \cos\theta \sin\theta + \cos^2\theta \tan\phi \right).
\end{equation*}

令$f(\theta) = \cos\theta \sin\theta + \cos^2\theta \tan\phi$,对 $\theta$ 求导:
\begin{equation*}
  f'(\theta) = \cos 2\theta - \sin 2\theta \tan\phi.
\end{equation*}

令 $f'(\theta) = 0$,得:
\begin{equation*}
  \cos 2\theta = \sin 2\theta \tan\phi \quad \Rightarrow \quad \tan 2\theta = \cot\phi.
\end{equation*}

因此:
\begin{equation*}
  2\theta = \frac{\pi}{2} - \phi \quad \Rightarrow \quad \theta = \frac{\pi}{4} - \frac{\phi}{2}.
\end{equation*}

为使石块沿坡面飞行距离最远,应以与水平方向夹角为$\theta = \frac{\pi}{4} - \frac{\phi}{2}$
(即 $\theta = 45^\circ - \dfrac{\phi}{2}$)抛出.

当 $\phi = 0$(平地),$\theta = 45^\circ$,符合经典结论.
 坡度 $\phi$ 越大,最优抛射角越小,应“低抛”以获得最大坡面射程.
\newline


\textbf{8 以椭圆一个焦点$F$为原点,沿半长轴方向设置极轴,椭圆的极坐标方程是$r=r_0/(1+e\cos\theta)$. 设所给椭圆的半长轴为$A$,半短轴为$B$,且$F$位于椭圆中心$O$的右侧,如图2所示,}
\begin{figure}[htbp]
  \centering
  \includegraphics[width=0.4\textwidth]{Fig/2.png}
  \caption{图2}
\end{figure}

\textbf{(a) 确定参量$r_0$, $e$与$A$,$B$的关系;}

椭圆的基本几何关系:
\begin{equation*}
  c = \sqrt{A^2 - B^2}, \quad e = \frac{c}{A} = \frac{\sqrt{A^2 - B^2}}{A}
\end{equation*}

在极坐标中,当 $\theta = 0$ 时,对应最近点(右顶点):
\begin{equation*}
  r_{\min} = A - c = \frac{r_0}{1 + e}
\end{equation*}

当 $\theta = \pi$ 时,对应最远点(左顶点):
\begin{equation*}
  r_{\max} = A + c = \frac{r_0}{1 - e}
\end{equation*}

联立解得:
\begin{equation*}
  A = \frac{r_0}{1 - e^2} \quad \Rightarrow \quad r_0 = A(1 - e^2)
\end{equation*}

由于 $e^2 = \dfrac{A^2 - B^2}{A^2}$,所以:
\begin{equation*}
  1 - e^2 = \frac{B^2}{A^2} \quad \Rightarrow \quad r_0 = A \cdot \frac{B^2}{A^2} = \frac{B^2}{A}
\end{equation*}

故,答案为:
\begin{equation*}
  \begin{aligned}
    e &= \frac{\sqrt{A^2 - B^2}}{A} \\
    r_0 &= \frac{B^2}{A}
  \end{aligned}
\end{equation*}

\textbf{(b) 若质点以$\theta=\omega t$方式沿椭圆运动,试导出$v_{\theta}$,$a_{\theta}$与质点角位置$\theta$的关系.}

已知 $\theta = \omega t$,故 $\dot{\theta} = \omega$,$\ddot{\theta} = 0$.

极坐标下速度与加速度的横向分量为:
\begin{equation*}
  v_\theta = r \dot{\theta}, \quad a_\theta = r \ddot{\theta} + 2 \dot{r} \dot{\theta} = 2 \dot{r} \omega
\end{equation*}

椭圆方程$r(\theta) = \frac{r_0}{1 + e \cos\theta}$.

则切向速度有:
\begin{equation*}
  v_\theta(\theta) = r \omega = \frac{\omega r_0}{1 + e \cos\theta} = \frac{\omega B^2}{A(1 + e \cos\theta)}
\end{equation*}

对于切向加速度,先求 $\dot{r}$:
\begin{equation*}
  \frac{dr}{d\theta} = \frac{r_0 e \sin\theta}{(1 + e \cos\theta)^2}
  \Rightarrow
  \dot{r} = \frac{dr}{d\theta} \cdot \omega = \omega \cdot \frac{r_0 e \sin\theta}{(1 + e \cos\theta)^2}
\end{equation*}

因此:
\begin{equation*}
  a_\theta(\theta) = 2 \omega \dot{r} = 2 \omega^2 \cdot \frac{r_0 e \sin\theta}{(1 + e \cos\theta)^2} = \frac{2 \omega^2 B^2 e \sin\theta}{A(1 + e \cos\theta)^2}
\end{equation*}

则,最终答案为:
\begin{equation*}
  \begin{aligned}
    v_\theta(\theta) &= \frac{\omega B^2}{A(1 + e \cos\theta)} \\
    a_\theta(\theta) &= \frac{2 \omega^2 B^2 e \sin\theta}{A(1 + e \cos\theta)^2}
  \end{aligned}
\end{equation*}
\newline


\textbf{9 极坐标系下的对数螺旋线可表述为$r=r_0e^{\alpha\theta}$,试用运动学方法导出曲率半径$\rho$.}

在平面曲线运动中,质点沿轨迹运动时,其加速度可分解为:
\begin{itemize}
    \item 切向加速度:$ a_t = \frac{dv}{dt} $
    \item 法向加速度:$ a_n = \frac{v^2}{\rho} $
\end{itemize}
其中 $ \rho $ 是曲率半径,正是我们要找的.

因此,如果我们能写出质点在该曲线上运动时的速度和加速度,并从中分离出法向加速度,就可以通过公式$a_n = \frac{v^2}{\rho} \quad \Rightarrow \quad \rho = \frac{v^2}{a_n}$求出曲率半径.

设质点在极坐标中运动,位置为 $ (r, \theta) $,则可求得速度和加速度表达式.

速度分量:
\begin{equation*}
  \begin{aligned}
    v_r &= \dot{r} \\
    v_\theta &= r\dot{\theta} \\
    v^2 &= \dot{r}^2 + (r\dot{\theta})^2
  \end{aligned}
\end{equation*}

加速度分量:
\begin{equation*}
  \begin{aligned}
    a_r &= \ddot{r} - r\dot{\theta}^2 \\
    a_\theta &= r\ddot{\theta} + 2\dot{r}\dot{\theta} \\
    a^2 &= a_r^2 + a_\theta^2
  \end{aligned}
\end{equation*}

但更关键的是,法向加速度是垂直于速度方向的,其大小为:
\begin{equation*}
  a_n = \sqrt{a^2 - a_t^2}
\end{equation*}

也可以直接从运动学关系得出:
\begin{equation*}
a_n = \frac{v^2}{\rho}
\end{equation*}

所以关键是计算 $ v^2 $ 和 $ a_n $.

我们令:
\begin{equation*}
  r(\theta) = r_0 e^{a\theta}
\end{equation*}

假设质点以某种方式运动,比如我们让 $ \theta = \omega t $,即匀角速运动,这样便于计算.但由于我们要求的是\textbf{几何性质}(曲率半径),它应与参数化无关,因此我们可以任意选择参数化方式.为了简化,不妨令 $ \theta = t $,即时间 $ t $ 等于角坐标 $ \theta $.

于是:
\begin{equation*}
  \theta = t \quad \Rightarrow \quad \dot{\theta} = 1, \quad \ddot{\theta} = 0
\end{equation*}

那么:
\begin{equation*}
  r = r_0 e^{a t} \quad \Rightarrow \quad \dot{r} = a r_0 e^{a t} = a r \quad \Rightarrow \quad \ddot{r} = a \dot{r} = a^2 r
\end{equation*}

因此,代入速度和加速度分量,得:
\begin{equation*}
  \begin{aligned}
    v_r &= \dot{r} = a r \\
    v_\theta &= r \dot{\theta} = r \cdot 1 = r \\
    v^2 &= v_r^2 + v_\theta^2 = (a r)^2 + r^2 = r^2(a^2 + 1) \\
    a_r &= \ddot{r} - r \dot{\theta}^2 = a^2 r - r \cdot 1 = r(a^2 - 1) \\
    a_\theta &= r \ddot{\theta} + 2 \dot{r} \dot{\theta} = 0 + 2(a r)(1) = 2a r
  \end{aligned}
\end{equation*}

总加速度大小平方为:
\begin{equation*}
  a^2 = a_r^2 + a_\theta^2 = [r(a^2 - 1)]^2 + (2a r)^2 = r^2[(a^2 - 1)^2 + 4a^2]
\end{equation*}

展开:
\begin{equation*}
  (a^2 - 1)^2 + 4a^2 = a^4 - 2a^2 + 1 + 4a^2 = a^4 + 2a^2 + 1 = (a^2 + 1)^2
\end{equation*}

所以:
\begin{equation*}
  a^2 = r^2 (a^2 + 1)^2 \quad \Rightarrow \quad a = r(a^2 + 1)
\end{equation*}

我们知道:
\begin{equation*}
  a^2 = a_t^2 + a_n^2
\end{equation*}

先求切向加速度 $ a_t $.它是速度大小的时间导数:
\begin{equation*}
  v = \sqrt{v^2} = r \sqrt{a^2 + 1}
\end{equation*}

因为 $ r = r_0 e^{a t} $,所以 $ v = r_0 e^{a t} \sqrt{a^2 + 1} $.

那么:
\begin{equation*}
  a_t = \frac{dv}{dt} = \frac{d}{dt} \left( r \sqrt{a^2 + 1} \right) = \dot{r} \sqrt{a^2 + 1} = a r \sqrt{a^2 + 1}
\end{equation*}

现在我们有:
\begin{itemize}
    \item $ v^2 = r^2 (a^2 + 1) $
    \item $ a_t = a r \sqrt{a^2 + 1} $
\end{itemize}

所以:
\begin{equation*}
  a_n^2 = a^2 - a_t^2 = [r^2 (a^2 + 1)^2] - [a^2 r^2 (a^2 + 1)] = r^2 (a^2 + 1)^2 (1 - a^2 / (a^2 + 1)) 
\end{equation*}

更简单地直接算:
\begin{equation*}
  a_n^2 = a^2 - a_t^2 = r^2(a^2 + 1)^2 - a^2 r^2 (a^2 + 1) = r^2 (a^2 + 1) \left[ (a^2 + 1) - a^2 \right] = r^2 (a^2 + 1) \cdot 1 = r^2 (a^2 + 1)
\end{equation*}

所以:
\begin{equation*}
  a_n = r \sqrt{a^2 + 1}
\end{equation*}

由:
\begin{equation*}
  a_n = \frac{v^2}{\rho} \quad \Rightarrow \quad \rho = \frac{v^2}{a_n}
\end{equation*}

我们有:
\begin{itemize}
    \item $ v^2 = r^2 (a^2 + 1) $
    \item $ a_n = r \sqrt{a^2 + 1} $
\end{itemize}

所以:
\begin{equation*}
  \rho = \frac{r^2 (a^2 + 1)}{r \sqrt{a^2 + 1}} = r \sqrt{a^2 + 1}
\end{equation*}

将 $ r = r_0 e^{a\theta} $ 代入,得:
\begin{equation*}
  \rho = r_0 e^{a\theta} \sqrt{a^2 + 1}
\end{equation*}

或写作:
\begin{equation*}
  \rho = r \sqrt{a^2 + 1}
\end{equation*}

综上所述,使用运动学方法,通过分析质点在对数螺旋线 $ r = r_0 e^{a\theta} $ 上的运动,计算其速度、加速度,分离出法向加速度,利用 $ a_n = v^2 / \rho $,得到曲率半径为:
\begin{equation*}
  \rho = r \sqrt{a^2 + 1}
\end{equation*}

\end{document}