\documentclass[aspectratio=169]{beamer}
\usepackage[UTF8]{ctex}
\usepackage{hyperref}

% other packages
\usepackage{latexsym,amsmath,xcolor,multicol,booktabs,calligra}
\usepackage{siunitx} % For \SI command
\usepackage{graphicx,pstricks,listings,stackengine}

\author{Yu Shu \& Chihao Shi}
\title{力学A(PHYS1001A.04):第一次习题课}
\subtitle{Course NOT easy: The Survival Guide 1}
\institute{School of Physics, USTC}
\date{Sept.27, 2025}
\usepackage{USTC}

% defs
\def\cmd#1{\texttt{\color{red}\footnotesize $\backslash$#1}}
\def\env#1{\texttt{\color{blue}\footnotesize #1}}
\definecolor{deepblue}{rgb}{0,0,0.5}
\definecolor{deepred}{rgb}{0.6,0,0}
\definecolor{deepgreen}{rgb}{0,0.5,0}
\definecolor{halfgray}{gray}{0.55}

\lstset{
    basicstyle=\ttfamily\small,
    keywordstyle=\bfseries\color{deepblue},
    emphstyle=\ttfamily\color{deepred},    % Custom highlighting style
    stringstyle=\color{deepgreen},
    numbers=left,
    numberstyle=\small\color{halfgray},
    rulesepcolor=\color{red!20!green!20!blue!20},
    frame=shadowbox,
}


\begin{document}
\kaishu

\begin{frame}
    \titlepage
    \begin{figure}[htpb]
        \begin{center}
            \includegraphics[width=0.15\linewidth]{pic/ustc_logo_fig-eps-converted-to.pdf}
        \end{center}
    \end{figure}
\end{frame}

\begin{frame}
    \tableofcontents[sectionstyle=show,subsectionstyle=show/shaded/hide,subsubsectionstyle=show/shaded/hide]
\end{frame}

\section{数学知识补充}

\subsection{行列式}

\begin{frame}
    “行列式” 除了在矢量叉乘的时候用得到,还有什么地方能用?
    行列式最初就是用来求解线性方程组的。
    矢量叉乘中的 “行列式” 只是借用了这个形式,并不是真正的行列式。

    下面以三阶行列式为例讲解其最基本的用法,对 n 阶行列式也适用。 
\end{frame}

\begin{frame}
    对于下述线性方程组:
    \begin{equation}
        \begin{cases}
            a_{11}x_1 + a_{12}x_2 + a_{13}x_3 = b_1 \\
            a_{21}x_1 + a_{22}x_2 + a_{23}x_3 = b_2 \\
            a_{31}x_1 + a_{32}x_2 + a_{33}x_3 = b_3
        \end{cases}
    \end{equation}
\end{frame}

\begin{frame}
    定义系数行列式:
    \begin{equation}
        D = \left |\begin{array}{ccc}
            a_{11} & a_{12} & a_{13} \\
            a_{21} & a_{22} & a_{23} \\
            a_{31} & a_{32} & a_{33} \\
            \end{array}\right|,
    \end{equation}
    \begin{equation}
        D_1 = \left |\begin{array}{ccc}
            b_{1} & a_{12} & a_{13} \\
            b_{2} & a_{22} & a_{23} \\
            b_{3} & a_{32} & a_{33} \\
            \end{array}\right|,
        D_2 = \left |\begin{array}{ccc}
            a_{11} & b_{1} & a_{13} \\
            a_{21} & b_{2} & a_{23} \\
            a_{31} & b_{3} & a_{33} \\
            \end{array}\right|,
        D_3 = \left |\begin{array}{ccc}
            a_{11} & a_{12} & b_{1} \\
            a_{21} & a_{22} & b_{2} \\
            a_{31} & a_{32} & b_{3} \\
            \end{array}\right|
    \end{equation}
\end{frame}

\begin{frame}
    则对于一般情况, 方程组 (1) 的唯一解为:
    \begin{equation}
        x_1 = \frac{D_1}{D},x_2 = \frac{D_2}{D},x_3 = \frac{D_3}{D}
    \end{equation}
    特别地,若 m = n = p = 0,该方程组被称为线性齐次方程组,
    其有非零解的充分必要条件为系数行列式 D = 0,且方程组 (1) 为无穷多组解。
    否则,方程组 (1) 只有唯一解。
\end{frame}

\subsection{矢量}

\begin{frame}{混合积}
    \begin{equation}
        (\vec{a}\times\vec{b})\cdot\vec{c} = (\vec{c}\times\vec{a})\cdot\vec{b} = (\vec{b}\times\vec{c})\cdot\vec{a}
    \end{equation}
\end{frame}

\begin{frame}{三重矢量积}
    计算方式: “先中间, 后外边”
    \begin{equation}
        (\vec{a}\times\vec{b})\times\vec{c} = (\vec{a}\cdot\vec{c})\vec{b} - (\vec{b}\cdot\vec{c})\vec{a}
    \end{equation}
\end{frame}

\subsection{导数与展开}

\begin{frame}{常见导数}
    \begin{equation}
        (a^x)' = a^x\ln a, (e^x)' = e^x
    \end{equation}
    \begin{equation}
        (\sin x)' = \cos x, (\cos x)' = -\sin x, (\tan x)' = \frac{1}{\cos^2 x}
    \end{equation}
    \begin{equation}
        (\log_a x)' = \frac{1}{x\ln a}, (\ln x)' = \frac{1}{x}
    \end{equation}
\end{frame}

\begin{frame}{反函数求导}
    \begin{equation}
        \frac{dy}{dx} = \frac{1}{\frac{dx}{dy}}
    \end{equation}
    e.g. $y = \arcsin x$,$x = \sin y$
    \begin{equation}
        (\arcsin x)' = \frac{1}{(\sin y)'} = \frac{1}{\cos y} = \frac{1}{\sqrt{1 - x^2}}
    \end{equation}
    \begin{equation}
        (\arccos x)' = -\frac{1}{\sqrt{1 - x^2}}, (\arctan x)' = \frac{1}{1 + x^2}
    \end{equation}
\end{frame}

\begin{frame}{复合函数求导}
    if $y = y(u)$, $u = u(x)$, then
    \begin{equation}
        \frac{dy}{dx} = \frac{dy}{du} \cdot \frac{du}{dx}
    \end{equation}
    e.g.  $y = x^x$, 记 $y = e^{x\ln x}$, $u(x) = x\ln x$, 则
    \begin{equation}
        \frac{dy}{dx} = \frac{dy}{du} \cdot \frac{du}{dx} = e^u \cdot (1\cdot\ln x+x\cdot\frac{1}{x}) = x^x(\ln x + 1)
    \end{equation}
\end{frame}

\begin{frame}{偏导数}
    多变量函数中对某个变量求偏导数式,其余变量均视为常数

    e.g. $r(x, y) = \sqrt{x^2 + y^2}$,$\theta(x, y) = \arctan\frac{y}{x}$
    \begin{equation}
        \begin{aligned}
            \frac{\partial r}{\partial x} &= \frac{1}{2}(x^2 + y^2)^{-\frac{1}{2}} \cdot 2x = \frac{x}{\sqrt{x^2 + y^2}} \\
            \frac{\partial r}{\partial y} &= \frac{y}{\sqrt{x^2 + y^2}} \\
            \frac{\partial \theta}{\partial x} &= \frac{1}{1+{(\frac{y}{x})}^2}\cdot(-\frac{y}{x^2}) = -\frac{y}{x^2 + y^2} \\
            \frac{\partial \theta}{\partial y} &= \frac{x}{x^2 + y^2}
        \end{aligned}
    \end{equation}
\end{frame}

\begin{frame}{矢量的导数}
    需要注意极坐标以及自然坐标系下的矢量对时间 $t$ 求导时,不能漏了基矢对时间的导数。

    e.g. 极坐标下
    \begin{equation}
        \begin{aligned}
            \frac{d\vec{e_r}}{dt} &= \frac{d\theta}{dt}\vec{e_\theta} \\
            \frac{d\vec{e_\theta}}{dt} &= -\frac{d\theta}{dt}\vec{e_r}
        \end{aligned} 
    \end{equation}
\end{frame}

\begin{frame}{微分}
    主要是积分中的换元操作需要使用。

    e.g. $y=f(x)$
    \begin{equation}
        dy = f'(x)dx
    \end{equation}
\end{frame}

\begin{frame}{常用Taylor展开}
    \begin{equation}
        \begin{aligned}
            (1+x)^n &= 1 + nx + \frac{1}{2!}n(n-1)x^2 + ... (x\rightarrow 0) \\
            e^x &= 1 + x + \frac{1}{2!}x^2 + ... (x\rightarrow 0) \\
            \ln(1+x) &= x - \frac{1}{2}x^2 + \frac{1}{3}x^3 - ... (x\rightarrow 0) \\
            \sin x &= x - \frac{1}{3!}x^3 + \frac{1}{5!}x^5 - ... (x\rightarrow 0) \\
            \cos x &= 1 - \frac{1}{2!}x^2 + \frac{1}{4!}x^4 - ... (x\rightarrow 0)
        \end{aligned}
    \end{equation}
\end{frame}

\subsection{积分}

\begin{frame}{补充内容:双曲函数}
    \begin{equation}
        \sinh x = \frac{e^x - e^{-x}}{2}, \cosh x = \frac{e^x + e^{-x}}{2}
    \end{equation}
    \begin{equation}
        \cosh^2 x - \sinh^2 x = 1
    \end{equation}
    \begin{equation}
        (\sinh x)' = \cosh x, (\cosh x)' = \sinh x
    \end{equation}
    \begin{equation}
        arsinh x = \ln(x + \sqrt{x^2 + 1}), arcosh x = \ln(x + \sqrt{x^2 - 1})
    \end{equation}
\end{frame}

\begin{frame}{基本内容}
    \begin{equation}
        \int \frac{1}{x}dx = \ln|x| + C, \int a^x dx = \frac{a^x}{\ln a} + C
    \end{equation}
    \begin{equation}
        \int \frac{1}{1 + x^2}dx = \arctan x + C, \int \frac{1}{\sqrt{1 - x^2}}dx = \frac{1}{2}\ln\left |\frac{1 + x}{1 - x}\right | + C
    \end{equation}
    \begin{equation}
        \int \frac{1}{\sqrt{1 - x^2}}dx = \arcsin x + C, \int \frac{1}{\sqrt{1 + x^2}}dx = \ln \left |x + \sqrt{1 + x^2}\right | + C
    \end{equation}
    \begin{equation}
        \int \frac{1}{\sqrt{x^2 - 1}}dx = \ln\left |x + \sqrt{x^2 - 1}\right | + C, \int \frac{dx}{\cos^2 x} = \tan x + C
    \end{equation}
    建议大家把熟练掌握上述常见的易混淆的积分,积分表里还有很多,有时间可以去推导一些。
    用到的方法主要就是两类换元法。
\end{frame}

\begin{frame}{第一类换元法: 凑微分}
    e.g.
    \begin{equation}
        \int \frac{dx}{\sqrt{x^2 -1}} = \int \frac{\frac{1}{2}d(x^2 - 1)}{\sqrt{x^2 - 1}} = \sqrt{x^2 - 1} + C
    \end{equation}
\end{frame}

\begin{frame}{第二类换元法}
    e.g.
    \begin{equation}
    \int_{-1}^{1} \sqrt{1 - x^2} dx
    \end{equation}
    令 $x = \sin t$,$t\in\left(-\frac{\pi}{2}, \frac{\pi}{2}\right)$(注意,第二类换元一定要注意新变量的取值范围),则原积分化为
    \begin{equation}
        \int_{-\frac{\pi}{2}}^{\frac{\pi}{2}} \cos^2 t \, dt = \int_{-\frac{\pi}{2}}^{\frac{\pi}{2}} \frac{\cos 2t + 1}{2} \, dt = \frac{\pi}{2}
    \end{equation}
\end{frame}

\begin{frame}{分部积分}
    \begin{equation}
        \int u(x)dv(x) = u(x)v(x) - \int v(x)du(x)
    \end{equation}
    e.g.
    \begin{equation}
        \int \ln(x)dx = x\ln(x) - \int xd\ln(x) = x\ln(x) - \int x\cdot\frac{1}{x}dx = x\ln(x) - x + C
    \end{equation}
    对于定积分, 同样也有
    \begin{equation}
        \int_{x_1}^{x_2} u(x)dv(x) = u(x)v(x)\Big|_{x_1}^{x_2} + \int_{x_1}^{x_2} v(x)du(x)
    \end{equation}
\end{frame}

\section{内容回顾与补充拓展}

\subsection{绪论}

\begin{frame}{宇宙的起源}
    \begin{figure}[htbp]
        \centering
        \includegraphics[width=0.5\textwidth]{pic/1.png}
        \caption{宇宙的起源}
    \end{figure}
\end{frame}

\begin{frame}{Planck Scale}
    \begin{equation}
        \begin{aligned}
            t_p &= \sqrt{\frac{\hbar G}{c^5}} \approx 5.4 \times 10^{-44} s \\
            l_p &= \sqrt{\frac{\hbar G}{c^3}} \approx 1.6 \times 10^{-35} m \\
            m_p &= \sqrt{\frac{\hbar c}{G}} \approx 2.2 \times 10^{-8} kg = 1.2 \times 10^{19} GeV
        \end{aligned}
    \end{equation}
    在自然单位制(Planck单位制)下均为。
\end{frame}

\begin{frame}{物理学的起源与发展}
    \begin{itemize}
        \item 希腊时期:亚里士多德
        \item 中世纪:大学的诞生与发展
        \item 牛顿之前:
        \begin{itemize}
            \item Roger Bacon:提倡经验主义,主张通过实验获得知识,倡导实验方法研究自然,重视对数学的研究。
            \item Nicolaus Copernicus:日心说
            \item Galileo Galilei:第一次把严密的逻辑推理、严格的数学论证和精密的实验测量结合在一起,使物理学成为真正科学。
        \end{itemize}
    \end{itemize}
\end{frame}

\begin{frame}{物理学的起源与发展}
    \begin{itemize}
        \item 牛顿力学:自然界第一个统一的范例
        \begin{itemize}
            \item 《自然哲学的数学原理》(1687),阐述了万有引力和三大运动定律,由此奠定现代物理学和天文学,并为现代工程学打下了基础
            \item 第一次实现了天体、地上、人间物体运动的统一;为太阳中心学说提供了强而有力的理论支持,是科学革命的一大代表
            \item 《光学》(1704)
            \item 在数学上,与莱布尼茨分享了发展出微积分学的荣誉
        \end{itemize}
        \item 经典物理学
        \begin{figure}[htbp]
            \centering
            \includegraphics[width=0.5\textwidth]{pic/2.png}
            \caption{经典物理学}
        \end{figure}
        \item 现代物理学:量子力学、相对论
    \end{itemize}
\end{frame}

\begin{frame}{补充拓展:量纲分析}
    量纲(dimension,dimension of a physical quantity)又称因次,是指物理量的基本性质和特征,它表示物理量与基本物理量(如长度、质量、时间、电流、温度、物质的量和光强度)的关系。
    量纲的表示通常使用大写字母,例如:长度($[L]$),质量($[M]$),温度($[\Theta]$),电流($[I]$),时间($[T]$),物质的量($[N]$),发光强度($[J]$)。
    这些基本量纲可以组合形成复合量纲。
    例如,速度的量纲是长度除以时间,表示为$[L][T]^{-1}$;加速度的量纲是长度除以时间的平方,表示为$[L][T]^{-2}$;力的量纲是质量乘以加速度,表示为$[M][L][T]^{-2}$。
\end{frame}

\begin{frame}{补充拓展:量纲分析}
    在国际单位制中有七个基本物理量,对应为七个基本量纲,则对于任意一个物理量,我们都可以写出以下的量纲式:
    \begin{equation}
        \dim{A} = [L]^{\alpha}[M]^{\beta}[\Theta]^{\gamma}[I]^{\delta}[T]^{\epsilon}[N]^{\zeta}[J]^{\eta}
    \end{equation}
    量纲指数为1的可以省略指数,指数为0的可以省略对应量纲;然而,当所有量纲指数皆为0时(称为无量纲),要将量纲记为“1”。
    值得注意的是,虽然物理量的量纲与取什么单位无关,但量纲却只有在一定的单位制下才有意义。
\end{frame}

\begin{frame}{补充拓展:量纲分析}
    对于不同物理量之间乘、除法导出新的物理量,量纲的计算满足数学上的指数计算法则,即:相乘则对应指数相加,相除则对应指数相减。
    例如,根据安培力计算公式$F=ILB$,可导出磁感应强度的量纲,有
    \begin{equation}
        \begin{aligned}
            \dim{B} &= \frac{(\dim{F})}{(\dim{I})(\dim{L})}\\
                    &= {\frac{[L][M][T]^{-2}}{[I][L]}} \\
                    &= [M][T]^{-2}[I]^{-1}
        \end{aligned}
    \end{equation}
\end{frame}

\begin{frame}{补充拓展:量纲分析}
    量纲服从的规律称为量纲法则,它有广泛的应用,一般只指出常用的两条:
    \begin{enumerate}
        \item 只有量纲相同的物理量,才能彼此相加、相减和相等;
        \item 指数函数、对数函数和三角函数的宗量应当是量纲1的。
    \end{enumerate}
    量纲法则是量纲分析的基础。
    若推出的公式不符合量纲法则,该式必然是错误的。
\end{frame}

\begin{frame}{补充拓展:量纲分析}
    $\pi$定理是由白金汉(E.Buckinghan)于1915年提出的一个定理,故又叫作白金汉定理。
    其内容为:设影响某现象的物理量数为n个,这些物理量的基本量纲为m个,则该物理现象可用N=n-m个独立的无量纲数群(准数)关系式表示。
\end{frame}

\begin{frame}{补充拓展:近似与极限}
    A simple pendulum consists of a point mass $m$ suspended by a massless string of length $l$. 
    When displaced by a small angle $\theta_0$ from its equilibrium position and released, it oscillates under gravity $g$. 
    Derive the equation of motion for the pendulum. 
    Explain how the small angle approximation ($\sin\theta \approx \theta$ for $\theta \ll 1$ radian) simplifies this equation. 
    What is the resulting expression for the period of oscillation $T$? 
    Discuss the physical significance of this approximation and the limit it implies.
\end{frame}

\begin{frame}{补充拓展:近似与极限}
    A point mass $m$ is launched vertically upward from the surface of the Earth (mass $M_\text{E}$, radius $R_\text{E}$) with an initial speed $v_0$. Assume no air resistance.

    (a) Using the approximation that the gravitational acceleration $g$ is constant, derive the maximum height $H_{\text{max, approx}}$ the object reaches.

    (b) Now, consider the exact gravitational force $F(r) = -G \frac{M_\text{E} m}{r^2}$, where $r$ is the distance from the Earth's center. Using conservation of energy, derive an expression for the exact maximum height $H_{\text{max, exact}}$.

    (c) Show that in the limit where the maximum height is much smaller than the Earth's radius ($H_{\text{max, exact}} \ll R_\text{E}$), the exact expression reduces to the approximate result found in part (a). Discuss the physical significance of this limit.
\end{frame}

\subsection{第一章:质点运动学}

\begin{frame}{Newton:绝对时空观}
    \begin{itemize}
        \item 时间间隔和空间间隔(长度)被认为是绝对量,是独立于所研究对象(物体)和运动而存在的客观实在
        \item 时间的流逝与空间位置无关,空间为欧几里德几何空间
        \item 任意一个物体对于不同的惯性坐标系的空间坐标量和时间坐标量之间满足伽利略变换
        \item 在弱引力、低速(远低于真空光速)运动情况下,绝对时空观符合实验结果
    \end{itemize}
    思考:牛顿为何要定义绝对时间和绝对空间?
\end{frame}

\begin{frame}{Newton:绝对时空观}
    \begin{multicols}{2}
        伽利略变换:
        \begin{equation}
            \begin{aligned}
                x' &= x - v t \\
                y' &= y \\
                z' &= z \\
                t' &= t
            \end{aligned}
        \end{equation}
        伽利略逆变换:
        \begin{equation}
            \begin{aligned}
                x &= x' + v t' \\
                y &= y' \\
                z &= z' \\
                t &= t'
            \end{aligned}
        \end{equation}
    \end{multicols}
\end{frame}

\begin{frame}{补充拓展:马赫原理}
    \begin{itemize}
        \item 物理的运动不是在绝对空间中的绝对运动,而是相对于宇宙中其他的物体的相对运动,所以不仅速度是相对的,加速度也是相对的。
        \item 在非惯性系中的物体所受的惯性力也是引力的表现,是宇宙中其他物体对该物体的总作用。
        \item 物体的惯性不是物体自身的属性,而是宇宙中其他物体作用的总结果。
    \end{itemize}
\end{frame}

\begin{frame}{爱因斯坦:相对时空观}
    \begin{itemize}
        \item 不同惯性系的时间-空间坐标之间不再遵从伽利略变换,而是遵从洛伦兹变换
        \item 时间、空间不是完全独立的, 时间-空间间隔(时空间隔)是不变量
        \item 时间间隔和空间间隔是相对的
    \end{itemize}
\end{frame}

\begin{frame}{爱因斯坦:相对时空观}
    \begin{multicols}{2}
        洛伦兹变换:
        \begin{equation}
            \begin{aligned}
                x' &= \gamma (x - v t) \\
                y' &= y \\
                z' &= z \\
                t' &= \gamma \left(t - \frac{v x}{c^2} \right)
            \end{aligned}
        \end{equation}
        洛伦兹逆变换:
        \begin{equation}
            \begin{aligned}
                x &= \gamma (x' + v t') \\
                y &= y' \\
                z &= z' \\
                t &= \gamma \left(t' + \frac{v x'}{c^2} \right)
            \end{aligned}
        \end{equation}
    \end{multicols}
    其中,$\gamma = \frac{1}{\sqrt{1 - \frac{v^2}{c^2}}}$ 是洛伦兹因子。
\end{frame}

\begin{frame}{补充拓展:牛顿水桶实验}
    \begin{enumerate}
        \item 桶吊在一根长绳上,将桶旋转多次而使绳拧紧,然后盛水并使桶与水静止,此时水是平面的。
        \item 接着松开,因长绳的扭力使桶旋转,起初,桶在旋转而桶内的水并没有跟着一起旋转,水还是平面的。
        \item 转过一段时间,因桶的摩檫力带动水一起旋转,水就形成了凹面。直到水与桶的转速一致。这时,水和桶之间是相对静止的,相对于桶,水是不转动的。但水面却仍然呈凹伏,中心低,桶边高。
    \end{enumerate}
    \begin{figure}[htbp]
        \centering
        \includegraphics[width=0.5\textwidth]{pic/3.png}
        \caption{牛顿水桶实验}
    \end{figure}
\end{frame}

\begin{frame}{补充阅读:牛顿水桶实验}
    很明显,水面的凹平与否跟水与桶地相对运动无关。例如(1)和(3)中,水与桶之间都没有相对运动,但是前者水面是平的,而后者水面却是凹的。
    水面的凹陷是由于受到了惯性离心力的结果。
    因此,惯性离心力的出现与水相对于桶的转动无关,那它与什么有关呢?
\end{frame}

\begin{frame}{补充阅读:牛顿水桶实验}
    \begin{itemize}
        \item 牛顿的理解:与绝对空间有关,惯性离心力产生于水相对于绝对空间的转动。因此是转动是绝对的,只有相对于绝对空间的加速才是真加速,才会有惯性力。因此牛顿认为,水桶实验证明了绝对空间的存在。
        \item 马赫的理解:马赫认为所有的运动都是相对的,根本没有绝对空间。因此,转动也不是绝对的,而是相对的。产生惯性力是因为水相对于全宇宙物质转动的结果。
        \item 爱因斯坦的理解: 爱因斯坦认为自己的广义相对论满足马赫原理,但是后来的深入研究表明,广义相对论与马赫原理并不一致。也就是说,惯性力的起源问题至今还没有搞清楚,牛顿桶实验所揭示的疑难至今仍然存在。
    \end{itemize}
\end{frame}

\begin{frame}{时间的度量}
    1967年10月在第十三届国际度量衡会议上规定:位于海平面上的铯原子的基态的两个超精细能级在零磁场中跃迁辐射的周期T与1秒的关系为\textbf{1秒 = 9,192,631,770 𝑇},这样的时间标准称为原子时。
    
    原子能级的典型特点是其稳定性、普适性和可重复性!所有的同类原子,其能级结构都完全相同,与测量哪个原子无关,与测量环境无关,与测量时间无关。

    常见:铯钟(基本定义)、氢钟(最稳定,参考定义)、光钟(最精确,参考定义,有望成为下一代基本定义)等。

    搜索资料:原子钟如何工作?
\end{frame}

\begin{frame}{空间的度量}
    1983年10月召开的第十七届国际计量大会上已正式通过了新的米的定义,即用光速值来定义“米”:光在真空中在1/299792458秒的时间间隔内所传播的路程长度。
\end{frame}

\begin{frame}{补充拓展:Heisenberg不确定性原理}
    海森堡不确定性原理(Heisenberg Uncertainty Principle)是量子力学的一个 fundamental principle,由德国物理学家维尔纳·海森堡于1927年提出。
    它揭示了微观粒子行为的本质特征,即某些成对的物理量不可能同时被精确测量。

    最常见和最重要的表述是关于粒子的 \textbf{位置} $x$ 和 \textbf{动量} $p_x$(沿$x$轴方向)的不确定度之间的关系:
    \begin{equation} \label{eq:uncertainty_principle}
        \Delta x \cdot \Delta p_x \geq \frac{\hbar}{2}
    \end{equation}
    其中:
    \begin{itemize}
        \item $\Delta x$ 是粒子位置的不确定度(标准差),$\Delta p_x$ 是粒子动量的不确定度(标准差)。
        \item $\hbar$ 是约化普朗克常数(reduced Planck constant),$\hbar = \frac{h}{2\pi}$。$h$ 是普朗克常数,其值约为 $h \approx \SI{6.626e-34}{\joule\second}$。因此,$\hbar \approx \SI{1.055e-34}{\joule\second}$。
    \end{itemize}
    这个不等式表明,粒子位置的不确定度 $\Delta x$ 和动量的不确定度 $\Delta p_x$ 的乘积有一个最小值 $\frac{\hbar}{2}$。
    这意味着,我们无法同时将一个粒子的位置和动量测量得任意精确。位置测得越精确($\Delta x$ 越小),动量的不确定度就越大($\Delta p_x$ 越大),反之亦然。
\end{frame}

\begin{frame}{补充拓展:Heisenberg不确定性原理}
    其他不确定性共轭对:
    能量和时间:
    \begin{equation}
        \Delta E \cdot \Delta t \gtrsim \frac{\hbar}{2}
    \end{equation}
    角动量的不同分量:
    \begin{equation}
        \Delta L_x \cdot \Delta L_y \geq \frac{\hbar}{2} \left| \langle L_z \rangle \right|
    \end{equation}
\end{frame}

\begin{frame}{芝诺佯谬}
    芝诺的佯谬,来源于芝诺时的局限性,芝诺时不可能度量阿基里斯追上乌龟之后的现象;并且采用了两种不同的时间度量

    “时间”与“时间的度量”不同,根据时间度量选取不同,某种时间的度量达到无限之后,还是可以有时间的

    度量的选取和统一非常重要
\end{frame}

\begin{frame}{运动的描述:模型和参考系}
    \begin{multicols}{2}
        参考系(Frame of reference):为确定物体的位置和描述物体运动而被选作参考的物体或物体群
        \begin{itemize}
            \item 定量描述对象运动需要
            \item 运动只能理解为物体的相对运动。在力学中,一般讲到运动,总是意味着相对于坐标系的运动。 ——Einstein
        \end{itemize}

        模型(Model)
        \begin{itemize}
            \item 保留在问题中起决定、主要作用的某些性质
            \item 撇开在问题中只起偶然作用或不起什么实质作用的性质
            \item 将复杂的实际对象简化成理想化的对象
        \end{itemize}
    \end{multicols}
\end{frame}

\begin{frame}{质点}
    质点(point mass):牛顿力学的基本模型。
    \begin{itemize}
        \item 物体不变形,不作转动(此时物体上各点的速度及加速度都相同,物体上任一点可以代表整个物体的运动),即物体平动
        \item 当物体运动的尺度远远大于其线度(此时物体的变形及转动显得并不重要)
    \end{itemize}

    在需要考虑多个质点的动力学问题时,通常把多个质点看做整体,即质点系。

    思考:为什么要求不作转动?
\end{frame}

\begin{frame}{坐标系与矢量描述}
    常用的坐标系:
    \begin{itemize}
        \item 直角坐标系$(x, y, z)$
        \item 球坐标系$(r, \theta, \phi)$
        \item 柱坐标系$(\rho, \phi, z)$
        \item 自然坐标系
    \end{itemize}

    矢量描述法
\end{frame}

\begin{frame}{运动方程与轨迹}
    运动方程( equation of motion):质点的位置关于时间的函数称为运动方程。

    轨迹( trajectory):质点在运动过程中描出的曲线方程, 将运动方程中的时间项消去,可以得到质点运动的轨道方程。
\end{frame}

\begin{frame}{基本运动学物理量}
    位移( displacement):质点在一段时间内位置的改变,是由质点初始位置引向末了位置的矢量。
    位移是位矢的增量,是时间和时间间隔的函数。
    与位矢不同,一旦参考系选定了,位移就与参考点的选择无关了。

    路程( distance):质点实际运动轨迹的长度叫路程。

    位移$\Delta \mathbf{r}$与路程$\Delta s$的区别:
    \begin{itemize}
        \item 前者是矢量,后者是标量
        \item $\Delta \mathbf{r}$既有大小变化又有方向变化,$\Delta s$只有大小的变化
        \item 通常$\Delta s \neq |\Delta \mathbf{r}|$,只有在单向直线运动中$\Delta s = |\Delta \mathbf{r}|$
        \item 当$\Delta t \to 0$时,$\lim_{\Delta t \to 0} |\Delta \vec{r}|\equiv |d\vec{r}| = ds \equiv \lim_{\Delta t \to 0} |\Delta s|$
    \end{itemize}
\end{frame}

\begin{frame}{基本运动学物理量}
    在直角坐标系中:$ds^2 = dx^2 + dy^2 + dz^2$

    空间弧长的微元:$|d\vec{r}| = ds = \sqrt{dx^2 + dy^2 + dz^2}$

    思考:有限的弧长如何计算?

    思考:是否有$dr = |d\vec{r}|$?物理含义一样吗?
\end{frame}

\begin{frame}{基本运动学物理量}
    速度(velocity):描述质点运动快慢和方向

    速率(speed):单位时间内经过的路程

    瞬时速度(instantaneous velocity,简称速度):位矢随时间的变化率

    瞬时速率(简称速率)

    平均速度和平均速率在数值上一般是不同的,但瞬时速率却等于瞬时速度的绝对值
\end{frame}

\begin{frame}{基本运动学物理量}
    加速度(acceleration):质点速度随时间的变化率

    瞬时加速度
\end{frame}

\begin{frame}{平面极坐标系}
    取平面内固定于参考物的一点O为原点(极点), 从它出发引出一条有刻度的射线为极轴,即建立起平面极坐标系(Polar coordinate system)。

    $r$是质点所在位置与极点间的距离,$\theta$是质点的连线与极轴的夹角,表示质点相对于极轴的方位。

    质点的位矢只有径向$\vec{r} = r \hat{r}$。

    单位矢量记为:$\vec{e}_r = \hat{r}$,$\vec{e}_\theta = \hat{\theta}$。
    \begin{equation}
        \begin{bmatrix}
            \hat{r} \\
            \hat{\theta}
        \end{bmatrix} = \begin{bmatrix}
            \cos \theta & -\sin \theta \\
            \sin \theta & \cos \theta
        \end{bmatrix} \begin{bmatrix}
            \hat{x} \\
            \hat{y}
        \end{bmatrix}
    \end{equation}
\end{frame}

\begin{frame}{平面极坐标系}
    \begin{figure}[htbp]
        \centering
        \includegraphics[width=0.95\textwidth]{pic/4.png}
    \end{figure}
\end{frame}

\begin{frame}{平面极坐标系}
    速度:$v_r = \dot{r}$,$v_\theta = r\dot{\theta}$

    加速度:$a_r = \ddot{r} - r\dot{\theta}^2$,$a_\theta = 2\dot{r}\dot{\theta} + r\ddot{\theta}$

    思考:以下两种特殊(极限)情况是何种运动形式?(1)$\dot{r}=0$,(2)$\dot{\theta}=0$

    思考:为何$a_{\theta}$里有个系数2?有何物理图像?$a_{\theta}$可以看成是切向加速度吗?
\end{frame}

\begin{frame}{轨道方程}
    \begin{figure}[htbp]
        \centering
        \includegraphics[width=0.95\textwidth]{pic/5.png}
    \end{figure}
\end{frame}

\begin{frame}{圆锥曲线的极坐标方程}
    \begin{figure}[htbp]
        \centering
        \includegraphics[width=0.95\textwidth]{pic/8.png}
    \end{figure}
\end{frame}

\begin{frame}{圆周运动}
    \begin{figure}[htbp]
        \centering
        \includegraphics[width=0.95\textwidth]{pic/6.png}
    \end{figure}
\end{frame}

\begin{frame}{圆周运动}
    \begin{figure}[htbp]
        \centering
        \includegraphics[width=0.95\textwidth]{pic/7.png}
    \end{figure}
\end{frame}

\section{学习经验分享}

\begin{frame}
    \begin{center}
        {\Huge \kaishu 学习经验分享}
    \end{center}
\end{frame}

\section{Q\&A}

\begin{frame}
    \begin{center}
        {\Huge\calligra Q\&A}
    \end{center}
\end{frame}

\begin{frame}
    \begin{center}
        {\Huge\calligra Thanks!}
    \end{center}
\end{frame}

\end{document}