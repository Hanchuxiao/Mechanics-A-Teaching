\documentclass[12pt]{article}
\usepackage{ctex} % 中文支持
\usepackage{amsmath}
\usepackage{amssymb}
\usepackage{bm} % 用于向量和张量的粗体数学符号
\usepackage{geometry}
\usepackage{xcolor}
\usepackage[colorlinks=true, linkcolor=blue]{hyperref}

\geometry{a4paper, left=2.5cm, right=2.5cm, top=2.5cm, bottom=2.5cm}

\definecolor{sectioncolor}{rgb}{0.2, 0.4, 0.6}
\usepackage{titlesec}
\titleformat{\section}{\Large\bfseries\color{sectioncolor}}{\thesection}{1em}{}
\titleformat{\subsection}{\large\bfseries}{\thesubsection}{1em}{}
\titleformat{\subsubsection}{\bfseries}{\thesubsubsection}{1em}{}

\title{\Huge\bfseries 理论力学问题深度解析}
\author{针对 PB25000095, PB25000081, PB25511991 同学问题的解答}
\date{\today}

\begin{document}

\maketitle

\section{问题五 (吕峻煕 PB25000095): 质点与质点系中动能定理与动量定理的关系}

\textbf{问题:} 为什么质点的动能定理和动量定理实际上是一个方程, 而质点系中就是相互独立的方程了?

\textbf{解答:} 这是一个非常深刻的问题, 触及了从单个质点过渡到质点系时, 力学规律如何演化的核心. 您的观察是正确的, 结论可以总结如下:
\begin{itemize}
    \item \textbf{对于单个质点:} 动能定理可以由动量定理 (牛顿第二定律) 通过数学推导得出, 因此它们在信息上是等价的, 不是相互独立的.
    \item \textbf{对于质点系:} 质心 (总动量) 定理和动能定理是\textbf{两个相互独立的定理}. 它们描述了系统不同层面的动力学行为, 一个不能从另一个推导出来.
\end{itemize}
下面我们来详细阐述其原因.

\subsection{对于单个质点}
质点的动量定理就是牛顿第二定律:
\begin{equation}
    \vec{F} = \frac{d\vec{p}}{dt} = m\frac{d\vec{v}}{dt} \quad (\text{动量定理})
\end{equation}
为了得到动能定理, 我们在该方程两边同时点乘质点的位移微元 $d\vec{r} = \vec{v}dt$:
\begin{equation}
    \vec{F} \cdot d\vec{r} = m\frac{d\vec{v}}{dt} \cdot \vec{v}dt
\end{equation}
我们知道, 合外力做的元功 $dW = \vec{F} \cdot d\vec{r}$. 对于右边, 我们有:
\begin{equation}
    m\vec{v} \cdot d\vec{v} = m(v_x dv_x + v_y dv_y + v_z dv_z) = \frac{1}{2}m d(v_x^2+v_y^2+v_z^2) = d\left(\frac{1}{2}mv^2\right) = dE_k
\end{equation}
所以, (2) 式就变成了 $dW = dE_k$. 这就是动能定理的微分形式.
\textbf{结论:} 对于单个质点, 动能定理仅仅是动量定理在位移路径上进行积分的结果. 它们描述的是同一个物理事实的不同侧面 (一个是瞬时关系, 一个是过程累积关系), 在数学上是直接关联的, 故不是独立的.

\subsection{对于质点系}
对于一个由 $N$ 个质点组成的系统, 情况发生了根本性的变化.
\begin{itemize}
    \item \textbf{质心定理 (总动量定理):} 它只关心\textbf{外力}与\textbf{质心运动}的关系.
        \begin{equation}
            \vec{F}_{\text{ext}} = \sum_i \vec{F}_{i, \text{ext}} = \frac{d\vec{P}_{\text{sys}}}{dt} = M_{\text{total}}\vec{a}_{CM} \quad (\text{质心定理})
        \end{equation}
        这个定理的推导用到了牛顿第三定律, 使得所有复杂的\textbf{内力都相互抵消}了. 它将整个系统“打包”成一个位于质心的质点来处理, 完全忽略了系统内部的结构和运动.

    \item \textbf{动能定理:} 它关心的是\textbf{所有力 (包括外力和内力)} 对系统\textbf{总动能} (包含了质心平动动能和内部动能) 的贡献.
        \begin{equation}
            W_{\text{total}} = W_{\text{ext}} + W_{\text{int}} = \Delta E_{k, \text{sys}} = \Delta (E_{k, CM} + E_{k, internal}) \quad (\text{动能定理})
        \end{equation}
\end{itemize}
\textbf{为什么它们是独立的?}
最关键的区别在于\textbf{内力 (Internal Forces)}的角色.
\begin{itemize}
    \item 在质心定理中, 内力\textbf{完全没有影响}. 无论系统内部发生爆炸还是缓慢的弹簧振荡, 只要没有外力, 质心就保持匀速直线运动.
    \item 在动能定理中, 内力\textbf{起着至关重要的作用}. 内力做功 $W_{\text{int}}$ 可以极大地改变系统的总动能.
\end{itemize}
\textbf{范例: 一颗炸弹在空中爆炸.}
假设一颗炸弹在最高点速度为零时爆炸成两片.
\begin{itemize}
    \item \textbf{质心定理视角:} 爆炸是内力作用. 在爆炸瞬间, 如果忽略空气阻力, 系统所受外力只有重力. 因此, 炸弹的质心将继续沿着原来的抛物线轨迹运动, 仿佛没有发生爆炸一样.
    \item \textbf{动能定理视角:} 爆炸前, 系统动能为零. 爆炸是化学内力做正功, 将化学能转化为动能. 爆炸后, 碎片的总动能急剧增加. $\Delta E_k = W_{\text{int}} > 0$.
\end{itemize}
在这个例子中, 质心运动 (总动量) 和系统总动能的演化完全是两回事. 你无法从质心定理推导出动能的变化, 也无法从动能定理推导出质心运动的规律. 因此, 对于质点系, 它们是两个描述系统不同方面动力学行为的、\textbf{相互独立的定理}.

\section{问题六 (冯子禾 PB25000081): 卢瑟福散射例题中的 $\sin\alpha$}
\textbf{问题:} 做题时常用到两种“有效势能”, 形式上大相径庭, 如何从更高层次理解它们?

\textbf{解答:} 这是一个非常精彩的问题, 触及了理论力学中一个核心且优雅的思想. 这两种有效势能的最终目的都是相同的: \textbf{将一个多维问题降维, 转化为一个等效的一维径向问题, 以便分析}. 它们形式不同, 是因为它们“打包”和“处理”角向运动的方式不同, 这源于两个问题背后不同的物理约束和对称性.

\subsection{情况一: 旋转参考系与离心势能}
\begin{center}
$V_{\text{eff}}(r) = -\frac{1}{2}m\omega^2r^2$ (+ 其他真实势能)
\end{center}
\begin{itemize}
    \item \textbf{适用情景:} 物体在一个以\textbf{恒定角速度 $\omega$}旋转的系统上运动. (例如: 旋转杆上的珠子, 旋转圆盘上的蚂蚁).
    \item \textbf{核心方法:} 切换到与系统一起旋转的\textbf{非惯性参考系}.
    \item \textbf{物理本质:} 在这个旋转系里, 原本复杂的二维螺旋运动被分解为简单的径向运动和静止的角向背景. 但为了抵消参考系加速带来的影响, 我们必须引入“虚拟”的\textbf{惯性力}, 其中最重要的就是沿径向向外的\textbf{离心力} $F_{cf} = m\omega^2r$.
    \item \textbf{势能的由来:} 离心力是一个只与位置 $r$ 有关的保守力 (在旋转系内). 我们可以为其定义一个势能, 通过积分 $F = -dV/dr$ 得到:
        \begin{equation}
            V_{cf}(r) = -\int F_{cf} dr = -\int m\omega^2 r dr = -\frac{1}{2}m\omega^2r^2
        \end{equation}
        这个 $V_{\text{eff}}$ (如果没有其他真实力的话) 就是所谓的\textbf{离心势能}. 它的本质是**对非惯性效应的能量化描述**. 此时, 物体在旋转系中的径向运动就可以用一个新的一维能量守恒来描述:
        \begin{equation}
            E' = \frac{1}{2}m\dot{r}^2 + V_{\text{eff}}(r) = \text{常量}
        \end{equation}
\end{itemize}

\subsection{情况二: 中心力场与角动量势垒}
\begin{center}
$V_{\text{eff}}(r) = \frac{L^2}{2mr^2} + U(r)$
\end{center}
\begin{itemize}
    \item \textbf{适用情景:} 物体在\textbf{中心力}作用下运动 (例如: 行星绕太阳, 电子绕原子核).
    \item \textbf{核心方法:} 在\textbf{惯性参考系}中分析, 并利用系统的对称性.
    \item \textbf{物理本质:} 中心力不产生力矩, 因此系统的\textbf{角动量 $L = mr^2\dot{\theta}$ 守恒}. 这是问题的关键对称性.
    \item \textbf{势能的由来:} 系统的总机械能 $E=T+U$ 在惯性系中是守恒的. 动能 $T$ 可以分解为径向和角向两部分:
        \begin{equation}
            E = \underbrace{\frac{1}{2}m\dot{r}^2}_{\text{径向动能}} + \underbrace{\frac{1}{2}mr^2\dot{\theta}^2}_{\text{角向动能}} + U(r) = \text{常量}
        \end{equation}
        为了得到一个只含 $r$ 和 $\dot{r}$ 的“一维”能量方程, 我们用角动量守恒 $L$ 来替换掉角向动能项:
        \begin{equation}
            \frac{1}{2}mr^2\dot{\theta}^2 = \frac{1}{2m}(mr^2\dot{\theta})^2\frac{1}{r^2} = \frac{L^2}{2mr^2}
        \end{equation}
        代回能量方程, 并把所有只依赖于 $r$ 的项归并到一起:
        \begin{equation}
            E = \frac{1}{2}m\dot{r}^2 + \left( \frac{L^2}{2mr^2} + U(r) \right)
        \end{equation}
        括号内的项就被定义为有效势能 $V_{\text{eff}}(r)$. 它的本质是\textbf{真实势能与角向动能的组合}. 其中的 $\frac{L^2}{2mr^2}$ 项被称为\textbf{离心势垒}或\textbf{角动量势垒}.
\end{itemize}

\subsection{更高层次的统一与对比}
\begin{center}
\begin{tabular}{|l|p{5.5cm}|p{5.5cm}|}
\hline
\textbf{特性} & \textbf{离心势能} & \textbf{角动量势垒} \\
\hline
\textbf{物理前提} & 角速度 $\omega$ \textbf{恒定} (外在约束) & 角动量 $L$ \textbf{守恒} (内在对称性) \\
\textbf{分析框架} & 非惯性系 & 惯性系 \\
\textbf{本质} & \textbf{惯性力}的能量化 & \textbf{角向动能}的能量化 \\
\textbf{来源} & 参考系的加速运动 & 系统的角动量守恒 \\
\textbf{拉格朗日视角} & 将约束 $\dot{\theta}=\omega$ 代入 $L$, 得到有效 $L_{\text{eff}}$ & 利用循环坐标 $\theta$ 得到守恒量 $p_\theta=L$, 简化径向方程 \\
\hline
\end{tabular}
\end{center}

**核心洞见:**
两种方法都是在“处理掉”角向自由度.
\begin{itemize}
    \item **旋转杆问题:** 角向运动不是一个真正的自由度, 它是一个由外部马达驱动的**运动学约束**. 我们通过切换到旋转系, 让这个约束变得“不可见”, 但代价是必须引入惯性力, 从而产生了离心势能.
    \item **中心力问题:** 角向运动是一个真正的自由度, 但由于系统的旋转对称性, 这个自由度的动力学行为极其简单 (角动量守恒). 我们利用这个守恒律作为“已知条件”, 将其从能量方程中\textbf{代换}出去, 从而把角向动能打包进了有效势能中.
\end{itemize}
因此, 这两种有效势能虽然形式不同, 但它们在简化问题上的哲学思想是相通的: \textbf{识别并分离出系统中简单的部分 (约束或对称性), 将其效应“能量化”并吸收到势能项中, 从而让我们能集中精力处理剩下更复杂的径向运动.}

\section{问题七 (胡少铠 PB25511991): 两种“有效势能”的深层理解}
\textbf{问题:} 助教好, 我对体积力和表面力的基本差异这一段有疑问, 体积力与空间点和时间有关没问题, 但是应力不应该不仅和空间点和受力面取向有关还和时间有关吗? 为什么这里没有说应力和时间有关?

\textbf{解答:}您提出了一个非常深刻且关键的问题, 您的直觉是完全正确的. \textbf{在最一般的情况下, 应力毫无疑问是与时间相关的.} 教材在该段落中省略了对应力时间依赖性的讨论, 这并非是物理上的错误, 而是一种在引入复杂新概念时非常常见的\textbf{“关注点分离”}的教学策略.

下面我们从三个层次来详细剖析这个问题.

\subsection{第一层: 教材的意图——强调“性质”上的根本差异}

该段落的核心教学目的, 不是给出现象最完备的数学描述, 而是要阐明\textbf{体积力 (Body Force)}和\textbf{表面力 (Surface Force), 即应力 (Stress)}在\textbf{“品格”或“内在属性”}上的根本不同.

\begin{itemize}
    \item \textbf{体积力的属性:} 在一个确定的时间 $t$ 和一个确定的空间点 $\vec{r}$, 体积力密度 (单位质量或单位体积所受的力) $\vec{b}$ 是一个\textbf{唯一的、确定的矢量}. 比如, 地球上某点的重力加速度 $\vec{g}$ 在某一时刻就是唯一确定的. 因此, 体积力形成了一个我们所熟悉的\textbf{矢量场} $\vec{b}(\vec{r}, t)$. 它的值只与时空点有关.

    \item \textbf{应力的属性:} 在一个确定的时间 $t$ 和一个确定的空间点 $\vec{r}$, 应力矢量 $\vec{p}$ \textbf{不是一个唯一的矢量}. 它的值还取决于你所考察的\textbf{微元面的朝向} (由单位法向量 $\vec{n}$ 描述). 在同一个点, 你可以切出无数个不同朝向的面, 每个面上感受到的应力矢量 $\vec{p}$ (包括大小和方向) 都可能不同.
\end{itemize}

\textbf{教学策略分析:}
作者在这里想要传达的最重要的、也是从标量/矢量场到张量场最关键的认知飞跃是: \textbf{应力不是一个简单的点函数, 它的定义本身就内含了对“方向”的依赖.}

为了突出这个最根本、最核心的新知识点, 作者暂时“冻结”了时间变量 $t$, 相当于在某个时间快照下进行讨论. 这样做的好处是可以把全部笔墨都用于解释应力对空间点 $\vec{r}$ 和法向量 $\vec{n}$ 这两个变量的依赖关系. 如果一开始就把时间 $t$ 也作为变量加进来, 可能会分散读者的注意力, 无法清晰地抓住这个从“矢量场”到“张量场”的概念升级.

\subsection{第二层: 完整的物理描述——应力张量}

您正确的直觉, 可以在更完整的物理描述中得到完美体现. 事实上, 一个点的应力状态是由一个\textbf{二阶张量}——\textbf{柯西应力张量 (Cauchy Stress Tensor)} $\boldsymbol{\sigma}$ 来完全描述的.

这个应力张量, 在最一般的情况下, 正是空间点和时间的函数:
\begin{equation}
    \boldsymbol{\sigma} = \boldsymbol{\sigma}(\vec{r}, t)
\end{equation}
它是一个 $3\times3$ 的矩阵, 包含了该时空点所有方向的应力信息. 而您书中提到的、作用在某个特定微元面 (法向为 $\vec{n}$) 上的应力矢量 $\vec{p}$, 是通过著名的\textbf{柯西应力定理 (Cauchy's Stress Theorem)}与应力张量联系起来的:
\begin{equation}
    \vec{p}(\vec{r}, \vec{n}, t) = \boldsymbol{\sigma}(\vec{r}, t) \cdot \vec{n}
\end{equation}
这个公式完美地回答了您的问题, 并统一了所有的依赖关系:
\begin{itemize}
    \item \textbf{物理状态的封装:} 在某个时空点 $(\vec{r}, t)$, 介质内部所有方向的力的相互作用状态, 都被“封装”进了应力张量 $\boldsymbol{\sigma}(\vec{r}, t)$ 这个数学对象中.
    \item \textbf{几何因素的分离:} 当我们具体考察某一个面时, 这个面的朝向作为一个纯粹的几何因素, 由单位法向量 $\vec{n}$ 来体现.
    \item \textbf{时间的体现:} 当流动是\textbf{非定常的 (unsteady)}时, 例如水管中的水锤效应, 或湍流中某点的压力脉动, 描述物理状态的应力张量 $\boldsymbol{\sigma}$ 就会随时间 $t$ 剧烈变化, 从而导致作用在任何一个面上的应力矢量 $\vec{p}$ 也随时间变化.
\end{itemize}

\subsection{第三层: 应力与时间相关的具体物理情景}

为了让这个概念更具体, 我们可以想象几个应力明显与时间相关的例子:
\begin{enumerate}
    \item \textbf{非定常流 (Unsteady Flow):} 想象一下, 您突然关闭一个快速流动的水管阀门. 一股压力波会以声速向上游传播. 此时, 在管道的某个固定点, 其压力 (法向应力的主要部分) 会在极短时间内发生剧烈变化. 这里的应力就是时间的函数 $\sigma_{xx}(x_0, t)$.
    \item \textbf{湍流 (Turbulence):} 在湍急的河流中, 如果您把手伸入水中某个固定位置, 会感觉到水流的冲击力忽大忽小, 方向也在不断变化. 这正是因为湍流中某点的速度、压力、粘性应力都在随时间做着快速、随机的脉动.
    \item \textbf{粘弹性材料 (Viscoelastic Materials):} 比如一块高分子凝胶. 您把它在 $t=0$ 时刻拉伸到一个固定长度并保持. 它内部抵抗形变的应力会随着时间慢慢减小, 这个现象叫做“应力松弛”. 这也是应力依赖于时间的典型例子.
\end{enumerate}

\subsection{总结}
\begin{itemize}
    \item 您的判断完全正确, \textbf{应力在一般情况下必然是时间的函数}.
    \item 教材在该段落中有意\textbf{省略时间 $t$}, 是一种\textbf{为了突出重点的、合理的教学策略}. 其目的是为了让读者集中精力理解应力与体积力最本质的区别: \textbf{应力在同一点的值依赖于受力面的取向, 而体积力则不依赖}. 这个性质是后续引入“张量”这一核心概念的物理基础.
    \item 最完整的描述是\textbf{应力张量 $\boldsymbol{\sigma}(\vec{r}, t)$}, 它清晰地表明了应力状态是时空点的函数. 您书中后续的章节, 尤其是在推导流体动力学基本方程 (如Navier-Stokes方程) 时, 必然会重新引入这种时间依赖性, 例如通过速度场的时间导数 $\partial \vec{v} / \partial t$ 来体现.
\end{itemize}

\end{document}