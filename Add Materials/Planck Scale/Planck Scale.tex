\documentclass[8pt,a4paper]{article}
\usepackage{ctex}
\usepackage{geometry}
\usepackage{amsmath}
\usepackage{hyperref}

\geometry{left=2.0cm,right=2.0cm,top=2.5cm,bottom=2.5cm}
\makeatletter
\def\@fnsymbol#1{\ensuremath{%
  \ifcase#1\or % 0
    \dagger\or % 1 - 改为 dagger
    *\or       % 2 - 原来的 dagger 变成 asterisk
    \ddagger\or % 3
    \S\or      % 4
    \P\or      % 5
    \|\or      % 6 (DOUBLE VERTICAL LINE)
    **\or      % 7
    \dagger\dagger\or % 8
    \ddagger\ddagger % 9
  \else
    \@ctrerr
  \fi
}}
\makeatother

\title{\Huge\textbf{拓展阅读:普朗克尺度}}
\author{TA 寒初晓\thanks{School of Physics, USTC, email:\url{shuyu2023@mail.ustc.edu.cn}}}
\date{Sep. 2025}

\begin{document}
\maketitle
\section{量纲}
量纲(dimension,dimension of a physical quantity)又称因次,是指物理量的基本性质和特征,它表示物理量与基本物理量(如长度、质量、时间、电流、温度、物质的量和光强度)的关系。
量纲的表示通常使用大写字母,例如:长度($\mathrm{L}$),质量($\mathrm{M}$),温度($\mathrm{\Theta}$),电流($\mathrm{I}$),时间($\mathrm{T}$),物质的量($\mathrm{N}$),发光强度($\mathrm{J}$)。
这些基本量纲可以组合形成复合量纲。
例如,速度的量纲是长度除以时间,表示为$LT^{-1}$;加速度的量纲是长度除以时间的平方,表示为$LT^{-2}$;力的量纲是质量乘以加速度,表示为$MLT^{-2}$。

在国际单位制中有七个基本物理量,对应为七个基本量纲,则对于任意一个物理量,我们都可以写出以下的量纲式:
\begin{equation}
\dim{A} = [L]^{\alpha}[M]^{\beta}[\Theta]^{\gamma}[I]^{\delta}[T]^{\epsilon}[N]^{\zeta}[J]^{\eta}
\end{equation}
$A$的量纲也可以表示为$[A]$。
量纲指数为1的可以省略指数,指数为0的可以省略对应量纲;然而,当所有量纲指数皆为0时(称为无量纲),要将量纲记为“1”。
值得注意的是,虽然物理量的量纲与取什么单位无关,但量纲却只有在一定的单位制下才有意义。

\subsection{量纲的乘除计算}
对于不同物理量之间乘、除法导出新的物理量,量纲的计算满足数学上的指数计算法则,即:相乘则对应指数相加,相除则对应指数相减。
例如,根据安培力计算公式$F=ILB$,可导出磁感应强度的量纲,有
\begin{equation}
\begin{aligned}
    \dim{B} &= \frac{(\dim{F})}{(\dim{I})(\dim{L})}\\
          &= {\frac{[L][M][T]^{-2}}{[I][L]}} \\
          &= [M][T]^{-2}[I]^{-1}
\end{aligned}
\end{equation}

\subsection{量纲法则}
量纲服从的规律称为量纲法则,它有广泛的应用,一般只指出常用的两条:
\begin{enumerate}
\item 只有量纲相同的物理量,才能彼此相加、相减和相等;
\item 指数函数、对数函数和三角函数的宗量应当是量纲1的。
\end{enumerate}
量纲法则是量纲分析的基础。
若推出的公式不符合量纲法则,该式必然是错误的。

\subsection{\texorpdfstring{$\pi$定理}{pi 定理 (Pi Theorem)}}
$\pi$定理是由白金汉(E.Buckinghan)于1915年提出的一个定理,故又叫作白金汉定理。
其内容为:设影响某现象的物理量数为n个,这些物理量的基本量纲为m个,则该物理现象可用N=n-m个独立的无量纲数群(准数)关系式表示。

例如,设在水平面上有一质量为m的物体,受一水平力F的作用加速滑动,加速度为a,物体与水平面之间的滑动摩擦因数为$\mu$,重力加速度大小为g。
则根据牛顿第二运动定律,可以写出以下关系式:
\begin{equation}
F - \mu mg = ma
\end{equation}
式中有5个物理量,涉及到$\rm{L}$、$\rm{M}$、$\rm{T}$三种基本量纲,因此根据$\pi$定理,可以写出2个无量纲数群关系式,即:
\begin{equation}
\frac{F}{ma}-\frac{\mu g}{a} = 1
\end{equation}
式中$\frac{F}{ma}$和$\frac{\mu g}{a}$均为无量纲数,常数1不作考虑。
于是,原来有五个未知量的式子就被转化为只有两个未知量的了。
实际应用当然会比这个复杂得多,然而原理是一样的。

\section{自然单位制}
在经典力学中,我们往往会使用国际单位制(SI),其中长度的单位是米(m),时间的单位是秒(s),质量的单位是千克(kg)。
有时,我们也会使用cgs单位制,此单位制下,长度的单位是厘米(cm),时间的单位是秒(s),质量的单位是克(g)。

然而,在高能物理和理论物理中,使用自然单位制(Natural Units)更为方便。
在自然单位制中,我们通常将光速 $c$ 和约化普朗克常数 $\hbar$ 设为1,即 $c = \hbar = 1$(事实上,有时也会取$\frac{1}{\alpha}$,此处$\alpha$是精细结构常数)。
与此同时,我们也会取其它数个常量为1,可选的常量有万有引力常数$G$、元电荷$e$、玻尔兹曼常数$k_B$等。
通过这种方式,我们可以简化许多物理方程,使得它们更易于处理和理解。
例如狭义相对论中的能量-动量关系式$E^2 = p^2c^2 + m^2c^4$在自然单位制下简化为$E^2 = p^2 + m^2$,大大简化了计算过程。 

下文我们列举几个常见的自然单位制。

\subsection{普朗克单位制}















\end{document}